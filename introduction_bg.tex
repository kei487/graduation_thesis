\chapter{序論}

\section{研究背景}
\subsection{ロボティクスに対する社会的要請}

\subsubsection{少子高齢化と労働生産性の課題}
21世紀に入り,先進諸国において少子高齢化に伴う生産年齢人口(15〜64歳の人口)
の減少が深刻な社会問題となっている.
特に日本においては,国立社会保障・人口問題研究所の推計によると,
生産年齢人口は1995年をピークに減少傾向にあり,
2070年には約4,500万人(2020年比で約4割減)まで落ち込むことが予測されている\cite{jfpe2023}.
この人口構造の変化は,経済成長の鈍化のみならず,
社会インフラの維持そのものを脅かす要因となっている.

物流業界においては,物流クライシスと呼ばれる状況が顕在化している.
電子商取引(E-commerce)の爆発的な普及により,小口配送の需要が急増している.
国土交通省の調査によれば,宅配便取扱個数は年間50億個(2023年度)を超え,
過去10年間で約1.3倍に増加している\cite{sthd2024}
トラックドライバーや倉庫内作業員の不足は慢性化しており,
労働環境の悪化と配送網の維持困難が懸念されている.
また,製造業の現場においても,熟練工の引退に伴う技術継承の問題や,
単純搬送作業への人員配置の困難さが指摘されている.

さらに,医療・介護の分野では,高齢者人口の増加に対し,
介護従事者の数は圧倒的に不足している.
厚生労働省の推計では,2040年度には約272万人の介護職員が必要とされているが,
現状のままでは数十万人規模の供給不足に陥る可能性が指摘されている\cite{nbnp2024}.
病院内での検体搬送,リネン類の回収,あるいは介護施設における見守りや配膳など,
定型的な業務の負担軽減は,強い需要がある.


\subsubsection{ロボットによる業務自動化と限界}
こうした労働力不足を補う手段として,
工場内物流においては,無人搬送車(Automated Guided Vehicle: AGV)の
導入が進められてきた.
AGVは,床面に敷設された磁気テープや反射テープ,
あるいは二次元コードといった物理的なガイドをセンサで読み取りながら走行する
ロボットである.
これにより,従来人の行っていた台車を押すような運搬作業をロボットが代替する
ことが可能になった.
AGVは環境が固定されており,かつタスクが定型的である場合に高い効率と
信頼性を発揮する.

一方で,AGVの導入と運用には,
物理的なガイドを必要とするという制約から
以下のような構造的な限界が存在すると指摘されている\cite{fragapane2021}.
これらの限界は,人や有人フォークリフトが頻繁に行き交う物流倉庫や,
一般の人々が存在する病院・商業施設といった
動的な障害物の多い環境へのロボット導入を阻む大きな障壁となっていた.

\begin{enumerate}
    \item \textbf{インフラ敷設コスト}: AGVに走行させたい経路のすべてにガイドを設置する必要があり,
	導入時の工事コストや期間が甚大である.
    \item \textbf{レイアウト変更の柔軟性欠如}: 製造ラインや倉庫のレイアウトを変更する際,
	ガイドの敷設し直しが必要となり,多大なコストとダウンタイムが発生する.
    \item \textbf{動的環境への非適応}: 想定された経路上に障害物が存在した場合,
	AGVはその場で停止することしかできず,回避して目的地へ向かうことができない.
\end{enumerate}


\subsubsection{AGVから自律移動ロボット(Autonomous Mobile Robot: AMR)へ}
AGVの課題を克服するため研究,開発が進められてきたのが,
自律移動ロボット(AMR)である.
AMRは,LiDAR(Light Detection and Ranging)やカメラといった外界センサ
を通じて周囲の環境の情報を得る.
このセンサ情報と事前に作成した環境地図を照らし合わせることで
地図内の自身の位置を推定し,ガイドを必要とせず走行する.

AMRは,AGVと違い,経路上に障害物を検知すると,回避経路を生成し,
その経路を追従することでタスクを継続することが可能である.
これは,AMRが走行する経路は,
物理的なガイドによる経路ではなく,ソフトウェア上の経路を走行するために,
動的に経路を変更することが可能であるため可能である.

また,物理的なガイドを必要としないため,導入時の工事が不要であり,
ソフトウェア上の設定変更のみで走行エリアや経路を変更できる.
この特性により,AMRは従来AGVが導入困難であった動的な環境への適用が進んでおり,
Society 5.0の中核を担う技術として期待されている\cite{stbc2016}.

\begin{figure}
  \centering
  \includegraphics[width=0.8\linewidth]{example-image-16x9.pdf}
  \caption{AGVとAMRの比較図}
\end{figure}


\subsubsection{自律移動ロボットの屋外への適用}
工場や倉庫といった屋内環境で培われたAMRの技術は,近年,より複雑かつ広範な屋外環境へとその適用範囲を拡大している.
特に,労働力不足が深刻な物流や農業分野において,屋外対応型AMRの実用化が急速に進展している.

物流分野では,配送拠点からエンドユーザーへの最終区間である
ラストワンマイルの配送コスト削減が最大の課題となっている.
人の代わりに荷物の配送を行う自律移動ロボットは,従来のトラック配送と比較して,
配送時間の短縮と環境負荷の低減を実現する有効な手段として位置付けられている\cite{alverhed2024}.
日本国内においても,2023年4月に施行された改正道路交通法により,
自律移動ロボットが遠隔操作型小型車として定義され,届出制による歩道走行が可能となった\cite{npa2023}.
これにより,パナソニックや楽天といった企業が住宅街での配送実証を行っており,
社会実装が進みつつある.

また,農業分野では,農林水産省がスマート農業を推進しており\cite{satp2024},
北海道大学の野口らによる先駆的な研究\cite{noguchi2011}を基盤として,
クボタやヤンマーなどの農機メーカーが有人監視下での自動走行(レベル2)および無人自動走行(レベル3)に対応した
ロボットトラクターを市場投入している\cite{maff2023}.
同様に,建設現場や鉱山といった過酷な環境においても,資材搬送や巡回監視を行うAMRの導入が進められている.

\begin{figure}
  \centering
  \includegraphics[width=0.8\linewidth]{example-image-16x9.pdf}
  \caption{楽天とかの自動配送ロボットの写真を入れたい}
\end{figure}

しかしながら,屋内環境と比較して,
屋外環境はロボットにとってタスクの遂行が困難となる要素が多い.
第一に,
ロボットの走る路面のロボットへ与える影響がある.
多くの移動ロボットは車輪型\cite{hara2025}であり,
平らで傾きのない屋内では,路面からロボットへ与える影響は小さい.
一方,屋外では,路面に段差や凹凸があり,
ロボットの姿勢の急激な変化や振動によるセンサノイズといった影響がある.
第二に,
天候や季節による路面状況の変化や,
時間変化による照明条件の変化がある.
雨,泥,雪,あるいは落ち葉などは,
風景の見た目を大きく変化させる.
また,直射日光による白飛びや逆光,夜間の低照度,
朝日,夕日による色温度の変化
といった光環境の変動は,
視覚情報の大きな変化を伴う.

これらの環境要因から
ロボットがどのように周囲を認識し,どのように自身の位置を知り,どのように経路を引くか
という課題が発生する.
次項では,
自律移動を実現するために現代のAMRがどのような技術要素によって構成されているか,
どのような課題が存在するか,
そのシステム概要について述べる.


\subsection{自律移動ロボットの技術構成}

\subsubsection{センシング技術}
ロボットが周囲の環境から情報を得るための技術である.

\begin{itemize}
    \item \textbf{LiDAR}: レーザー光を照射し,反射光が戻ってくるまでの時間(Time of Flight)
	や位相差から距離を計測する.
	北陽に代表される
	2次元平面をスキャンする2D LiDARが主流であったが,
	近年では,HesaiやLivoxといった中国メーカーによる,
	3次元点群を取得可能な3D LiDARの低価格化が進んでいる.
    \item \textbf{カメラ}: RGB画像に加え,深度情報を取得できるRGB-Dカメラや
	ステレオカメラが利用される.
	Visual SLAMや物体認識との親和性が高い.
    \item \textbf{オドメトリ(Odometry)}: 車輪の回転数(エンコーダ値)や
	IMU(慣性計測装置)のデータから,ロボットの相対的な移動量を推定する.
	短期的には高精度だが,累積誤差が生じるため単独では長距離移動に適さない.
    \item \textbf{GNSS(Global Navigation Satellite System)}: 
	アメリカのGPSや日本のQZSS(みちびき)といった衛星測位システムの総称.
	屋外環境において地球上での位置を取得する主要な手段となる.
	搬送波位相を用いるRTK-GNSSにより,数センチメートルの精度での測位が可能となり,
	配送や農業を行うロボットに広く採用されている.
    \item \textbf{無線通信 (WiFi/5G)}: アクセスポイントからの信号強度(RSSI)や
    電波の到達時間(RTT)を用いた測位に利用される.
\end{itemize}

\subsubsection{自己位置推定(Localization)技術}
ロボット自身が今どこにいるかを計算する技術は,
自律移動にとって欠かせない技術である.
オドメトリの累積誤差を補正し,地図座標系上での絶対位置を特定するために,
様々な手法が開発されてきた.


\paragraph{確率的ロボティクスの枠組み}
1990年代以降,Thrunらによって提唱された
確率的ロボティクス(Probabilistic Robotics)\cite{thrun2005}は,
センサのノイズや環境の不確実性を確率分布として扱うことで,ロバストな自己位置推定を可能にした.
代表的な手法として,(拡張)カルマンフィルタと
パーティクルフィルタ(Particle Filter)がある.
パーティクルフィルタの実装方法として広く使われているのが,
モンテカルロ位置推定(Monte Carlo Localization: MCL)\cite{Fox1999}
である.
これは,ロボットの存在確率分布を多数の粒子(パーティクル)で表現し,
センサ観測と動作モデルに基づいて粒子の重みと位置を更新する手法である.
MCLは,ロバスト性において他の手法より優れており,
現在のAMRのデファクトスタンダードとなっている.

%\dummyfig{パーティクルフィルタによる位置推定の概念図}{4cm}

\paragraph{SLAM (Simultaneous Localization and Mapping)}
地図を持たない未知の環境においては,自己位置推定と同時に環境地図の作成を行うSLAM技術が用いられる.
\begin{itemize}
    \item \textbf{LiDAR SLAM}: GMappingやCartographerなど,
	レーザーレンジファインダを用いたグリッドマップ生成手法.
  GLIMに代表されるGraph-based SLAMと呼ばれる手法は,
  ロボットの姿勢(ノード)と観測制約(エッジ)をグラフ構造として表現し,
  非線形最小二乗法によって全体最適化を行うもの.
    \item \textbf{Visual SLAM}: ORB-SLAMなど,カメラ画像の特徴点を用いて
	疎な地図(Feature Map)を作成する.テクスチャの豊富な環境に強い.
\end{itemize}


\paragraph{経路計画(Path Planning)技術}
自己位置推定技術の成熟により,
ロボットは環境内での自身の位置を推定できるようになった.
また,SLAM技術により,障害物の配置や通行可能な領域を表す
環境の地図を作成することも可能となった.


ロボットが実際にタスクを遂行するためには,自己位置推定だけでなく,
現在地(Start)から目的地(Goal)まで,
障害物を回避しつつ,かつ効率的(最短時間,最小エネルギーなど)
に到達するための軌道を決定しなければならない.


経路計画は,自己位置推定の結果と環境地図,目的地を入力とし,
ロボットのアクチュエータ(モータ)への制御指令を出力とする,
自己位置推定がいかに正確なものであっても,
経路計画が不適切であれば,ロボットは遠回りをするか,
狭い通路で立ち往生するか,最悪の場合は動的障害物と衝突する危険性がある.


n章で述べたように,社会実装が進むにつれて,ロボットが稼働する環境はより複雑化している.
静的な障害物だけでなく,人や他のロボットが移動する動的環境や,
路面に存在する微小な障害物によるノイズや,環境の変化によるセンサ計測の不確実性
の存在するような環境が挙げられる.
これらの要因を考慮し,安全かつ最適な経路をリアルタイムに導出することは,
自律移動ロボットの社会実装を進めるにあたって,重要な研究課題である.
次節では,この経路計画に関する従来研究を概観し,
本研究で扱う価値反復法の位置付けを明らかにする.
% ↑できるか??



