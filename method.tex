\chapter{提案手法}\label{chap:method}
%3章からだいぶ粗め
価値反復ROSパッケージにグラフ探索手法を組み合わせ,
グラフ探索手法で算出した経路から$V$に勾配を作ることで,
走り出しまでの時間を短縮する手法を提案する.
\ref{sec:with2dastar}章は,文献\cite{中村2024}で発表したものであり,
\ref{sec:with3dastar}章は,文献\cite{中村2025}で発表したものである.


\section{価値反復とA*を組み合わせた大域計画}\label{sec:with2dastar}
A*と価値反復を並列に計算し,A*の計算結果を価値反復に適用する.
既存の価値反復と
並行でA*探索を実行し,図\ref{fig:propose}のように,
A*の見つけた経路沿いに
$V$の値を書き換えるというものである.
図のようにロボットが行動を開始する状態から
目的地まで,$V$の値を少しずつ減らしていくように
書き換えることで,価値反復が経路を見つける前に,
ロボットを目的地に向かわせるようにする.


$V$の書き換えは,A*の見つけた経路沿いの各$s$に対して,
\begin{align}
	V(s) \longleftarrow K f(s) \label{eq:a-star_weight}
\end{align}
という代入の式を用いて行う.
$K$は定数であり,$f$は$s$から
目的地までの経路上での道のりである.

この方法では,ロボットと目的地の間の
環境が迷路のようになっていなければ,
A*で見つかる経路がほぼ最適な経路となり,
価値反復のみの場合よりもロボットが速やかに
目的地に向かえるようになる.
一方,迷路のような環境だと,
たとえばA*で見つけた経路が遠回りで,
そのあとで価値反復がよりよい経路を見つけると,
目的地までの時間が増えてしまう可能性がある.

また,A*がほぼ最適な経路を見つけられる場合,
価値反復は大域計画に対しては必須ではなくなる.
しかし,A*の見つけた経路の最適化や,
未知の障害物が現れた場合の$V$の修正や
迂回先の$V$の計算に必要となる.

ここで用いるA*は,$XY$平面を探索するものであり,
価値反復が探索する$XY\theta$空間に比べ,1つ次元が低い.
これにより,特に地図が大きくなったとき,計算時間が相対的に小さくなることが期待される.
その一方で,算出する経路が$XY\theta$において最適ではないものとなり,
$V$の勾配が思わぬ方向に傾くような悪影響が考えられる.

\section{3D A*の適用}\label{sec:with3dastar}
そこで,価値反復と同じ探索空間を持つA*(3D A*)を用いて
経路を算出し,用いる手法を提案する.

3D A*は,
価値反復と同様に,グリッドに加えて
$\theta$方向にも離散化し,グラフを構築する.
$XY$方向の遷移コストと$\theta$方向の遷移コストは単位が一致するよう
距離から時間に変更した.

ヒューリスティック関数$H$については,
\cite{nakamura2024}での平面上のユークリッド距離を計算するものから,
\begin{align}
	H(s) \triangleq W_1 \sqrt{  (x_\text{c} - x_\text{g})^2 + (y_\text{c} - y_\text{g})^2 }
	+ W_2 | \theta_\text{c} - \theta_\text{g} | \label{eq:3d-astar-heurisic}
\end{align}
に変更する.
ここで,
$(x_\text{c}, y_\text{c}, \theta_\text{c})$は各離散状態の中心の座標,
$(x_\text{g}, y_\text{g})$は目的地の中心地点の$xy$座標,
$\theta_\text{c} - \theta_\text{g}$は,$(x_\text{c}, y_\text{c}, \theta_\text{c})$
から見た$(x_\text{g}, y_\text{g})$の方角である.
$W_1, W_2$は定数である.
$W_1$と$W_2$の比は,ロボットが一定時間あたりに移動できる距離と方向転換できる角度の比で決まる.

