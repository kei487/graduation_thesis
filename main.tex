\documentclass[a4paper,11pt]{jsbook}

\newcommand{\V}[1]{\boldsymbol{#1}}
\def\thline{\noalign{\hrule height 1pt}}
\def\tvline{\vrule width 1pt}

\usepackage{here} %図の場所の指定で[H](ここに貼る)を指定するためのパッケージ
\usepackage{makeidx}
\usepackage{amsmath}
\usepackage{amssymb}
\usepackage[dvipdfmx]{graphicx} %dvipdfmxはjpgやpngの張り込みのために使用
\usepackage{url}
\usepackage{mwe}  % dummy fig用

\makeindex

\pagenumbering{roman}

\begin{document}

% 表紙
\title{令和7年度 卒業論文 \\
    自律移動ロボットのための価値反復と\\
    A*探索を組み合わせた大域経路計画 \\
}


\author{中村啓太郎\\
    China Institute of Technology}


\date{2026年1月29日}

\maketitle

%%% 但し書き等 %%%
% \clearpage

%%% 献辞 %%%
% D論, あるいは誰かを亡くしたときの卒論, 修論等で
% 配偶者や配偶者の予定となる人の名前は覚悟をもって書くこと
% \thispagestyle{empty}
% \vfil
% \ \\
% \vspace{15em}
% \begin{center}
% 	{\Large 最愛の京成線に捧ぐ }
% \end{center}
% 献辞をかかない場合はここまでコメントアウト

\chapter*{謝辞}\addcontentsline{toc}{chapter}{謝辞}

本論文は,千葉工業大学先進工学部未来ロボティクス学科
 自律ロボット研究室(上田隆一研究室)
において執筆したものです.

研究室では,多くのメンバーと共に過ごし,
様々な刺激をいただきました.
その中で,私の研究を手伝っていただくこともあり
感謝します.


修士2年の吉越さんには,研究で使用する
ロボットの不調のトラブルシューティング
の折に大変お世話になりました.
にっちもさっちも行かなくなった際には,
深い経験からアドバイスいただきありがとうございました.


修士2年の船井さん,
修士1年の茂さんと川原さんには,
よく研究室での話し相手になっていただきありがとうございました.
私が,よく研究に行き詰まると勝手に話しかけていましたが,
快くお相手してくださって幾度も助けられました.


また,修士1年の佐々木さんには,
ロボットの改修でお世話になりました.
ハードウェアの知識の足りない私にとって
よい学びとなりました.


修士1年の永木さんには,
同じく価値反復を用いた研究を行っていることから,
作成したシミュレータ環境を使わせていただきありがとうございました.


同級生である市東さん,鷲尾さんは,
よく研究室で無駄話に付き合ってくださりありがとうございました.
来年度以降の修士課程でもよろしくお願いします.


また,同級生の鈴木さん,藤野さんも,
よく話してくださいました.
これからのご活躍を祈っています.


学部3年の,辻さん,水牧さん,和田さん,
根本さん,平地さん,
は,私のくだらない話にも付き合ってくださり
ありがとうございました.
来年度の活躍を楽しみにしています.


最後に,
本研究に取り組み,2度の学会の予稿と本稿を執筆するにあたり, 
ご指導を頂いた上田隆一教授に感謝します.


\tableofcontents

%\cleardoublepage

%%% 本文 %%%
% 章のページの先頭は左側(奇数ページ)に来る

\cleardoublepage
\pagenumbering{arabic}

\chapter{序論}

\section{研究背景}
経路計画技術は,現代社会において必要性が見込まれる
自律移動システムの要素技術である.
少子高齢化によって労働人口が減少している問題に対して,
自律移動システムを持つロボットが,
解決に寄与すると考えられている.
ビルや商業施設の警備や,店舗や家屋の清掃,配膳,農業といった,
従来,人が行っていた作業は,移動を基礎とした作業である.
そのため,自律移動システムは,
これらの作業をロボットが代替する際に,要求される.


自律移動システムは,移動するロボットが,
人の操作を必要とせずに,安全に移動を行うシステムである.
ロボットが安全に移動するためには,
障害物を回避ないしは停止し,ぶつからないことが求められる.
障害物は,建造物やロボットが乗り越えられない段差といった
短い時間で位置の変化しない固定障害物と,
人間や停車している車,看板といった短い時間で位置の変化する移動障害物
の2種類に分けられる.

自律移動システムには,
大別して自己位置推定,経路計画の2つの要素技術がある.


自己位置推定は,ロボット自身が
環境からセンサを通じて得られる情報から
環境内の自身の位置を推定する技術である.
センサ情報から自身の位置を推定
する技術である.
には,MCL(Monte Carlo Localization)\cite{fox1999}
が用いられる.






現在,経路計画にはよくA*アルゴリズムが用いられ,
経路追従と障害物回避にはDWA(Dynamic Window Approch)が用いられる.
このように異なるアルゴリズムを用いることにより,計算が簡単になる反面
競合する可能性が残っている.
そのため,経路計画と障害物回避を同時に行う手法として,
ポテンシャル法がある.
でも局所最適に陥る問題があり,価値反復はそれを解決する.



\section{従来研究}
上田らは価値反復ROSパッケージを開発した.


\section{研究目的}
価値反復ROSパッケージを用いたナビゲーションにおいて,
走り出しまでの時間を短縮することで
移動にかかる時間を短縮することを目的とする.


\section{論文の構成}
章では,問題設定と価値反復アルゴリズム,A*アルゴリズムについて述べ,
章では,提案手法,章では,実験について述べる.
章でまとめる.


\chapter{移動ロボットの経路計画問題}\label{chap:purpose}

\section{扱う問題}
本研究では,移動ロボットを平面上で自律移動させる問題を扱う.
広く用いられている Nav2[?]%\cite{}] 
(ROS 2 の標準ナビゲーションパッケージ)が扱う問題と同様の
問題である.
ロボットは,行動を始めるタイミングで目的地の座標を与えられ,
障害物との衝突を避けながらできる限り短い時間で目的地まで
到達しなければならない.
ロボットが移動を行う空間を環境と言い,
図2%\ref{fig:map}
にその環境の例を示す.
環境には,世界座標系が 2 次元の直交座標系で設定されており,
ロボットは,位置$(x, y)$と,$x$軸となす角$\theta$を向きとして
持っており,これらをまとめてロボットの状態(位置と向き)$x = (x, y, \theta)$
3 変数で表現される.
目的地地点は図中の destination area のように $XY$
平面上の領域や,$xy\theta$空間内の領域として与えられる.
環境中には,その位置は既知である固定障害物と
移動障害物が存在するが,これらは VI パッケージの既存の機能で対処可能である.
しかし,本研究では移動障害物の回避は陽には扱わない.

\section{価値反復}
価値反復アルゴリズムは,ベルマン方程式の形式に乗るように
定式化することで,最適な状態価値関数を得られるアルゴリズムである.

\section{経路探索アルゴリズム}

\subsection{Dijsktra法}
よく知られたグラフ探索アルゴリズムである.

\subsection{A*アルゴリズム}
Dijkstra法を発展させたものであり,ゴールまでのコストを推定する
ヒューリスティック関数を追加したものである.

\chapter{提案手法}\label{chap:method}
%3章からだいぶ粗め
価値反復ROSパッケージにグラフ探索手法を組み合わせ,
グラフ探索手法で算出した経路から$V$に勾配を作ることで,
走り出しまでの時間を短縮する手法を提案する.
\ref{sec:with2dastar}節% @@@章じゃなくて節@@@
は,文献\cite{中村2024}で発表したものであり,
\ref{sec:with3dastar}節は,文献\cite{中村2025}で発表したものである.


\section{価値反復とA*を組み合わせた大域計画}\label{sec:with2dastar}
A*と価値反復を並列に計算し,A*の計算結果を価値反復に適用する.
既存の価値反復と
並行でA*探索を実行し,図\ref{fig:propose}のように,
A*の見つけた経路沿いに
$V$の値を書き換えるというものである.
図のようにロボットが行動を開始する状態から
目的地まで,$V$の値を少しずつ減らしていくように
書き換えることで,価値反復が経路を見つける前に,
ロボットを目的地に向かわせるようにする.


$V$の書き換えは,A*の見つけた経路沿いの各$s$に対して,
\begin{align}
	V(s) \longleftarrow K f(s) \label{eq:a-star_weight}
\end{align}
という代入の式を用いて行う.
$K$は定数であり,$f$は$s$から
目的地までの経路上での道のりである.

この方法では,ロボットと目的地の間の
環境が迷路のようになっていなければ,
A*で見つかる経路がほぼ最適な経路となり,
価値反復のみの場合よりもロボットが速やかに
目的地に向かえるようになる.
一方,迷路のような環境だと,
たとえばA*で見つけた経路が遠回りで,
そのあとで価値反復がよりよい経路を見つけると,
目的地までの時間が増えてしまう可能性がある.

また,A*がほぼ最適な経路を見つけられる場合,
価値反復は大域計画に対しては必須ではなくなる.
しかし,A*の見つけた経路の最適化や,
未知の障害物が現れた場合の$V$の修正や
迂回先の$V$の計算に必要となる.

ここで用いるA*は,$XY$平面を探索するものであり,
価値反復が探索する$XY\theta$空間に比べ,1つ次元が低い.
これにより,特に地図が大きくなったとき,次元の違いから
探索にかかる時間が価値反復の勾配の伝播に比べ小さくなることが期待される.
その一方で,算出する経路が$XY\theta$において最適ではないものとなり,
$V$の勾配が思わぬ方向に傾くような悪影響が考えられる.

\begin{figure}[htb]
  \centering
  \includegraphics[width=0.8\linewidth]{example-image-16x9.pdf}
  \caption{propose method with 2D A*}
\end{figure}


\section{3D A*の適用}\label{sec:with3dastar}
そこで,価値反復と同じ探索空間を持つA*(3D A*)を用いて
経路を算出し,用いる手法を提案する.
3D A*の算出する経路は,用いるヒューリスティック関数$H$が許容的にあるとき,
最適性が保証される.これにより,価値反復のコストと大きな差がない
暫定的なコストを計算できることが期待できる.
その一方で,当然ながら2D A*に比べて計算量は多くなる.
2D A*の算出した経路を$V$に適用による移動時間の増加量と
3D A*で増加した計算時間の増加量のどちらが多いのか,
非自明であり,検証する必要がある.


\chapter{実装}

\section{A*の実装}
A*の探索空間は,$xy$平面であり,


\section{価値反復とA*の併用}
A*による経路の価値反復への適用は,
次の式にしたがって行う.




\chapter{実験}\label{chap:experiment}


\section{シミュレータ実験}

提案手法の走り出しまでの時間を短縮する効果の有無を
検証するために,シミュレータを用いて実験を行う.
シミュレータ環境では,動的な障害物が存在せず,
障害物回避による移動時間の増加を考慮する必要がない
理想的な環境で効果を検証できる.


\subsection{実験条件}
シミュレータ実験では,
千葉工業大学津田沼キャンパスで得られた
図\ref{fig:map}の地図を使用する.
紙面横方向が300m,縦方向が200mの6haの環境の地図で,
形式は,ROSのナビゲーションスタックで用いられる
占有格子地図である.
白色が障害物のない画素,
黒色が障害物のある画素,灰色が不明な画素を表す.
この地図のパラメータは表\ref{tab:map}の通りである.

価値反復ROSパッケージでは,地図のグリッドに加え,
$\theta$方向にも離散化を行い$mathbb{S}$を構成する.
$S$の諸元を表\ref{tab:vi_s}に示す.
要素数は1億に達し,移動可能なエリアだけでも1千万に達する.

実験で使用するシミュレータは,
図\ref{fig:map}から作成したシミュレータを使用する.
シミュレータには,ROSの標準的なシミュレータである
Gazeboを使用し,
シミュレータ内で用いる静的な障害物の作成には,
占有格子地図から作成するプログラム\cite{ikebe2020sim}
を使用した.

シミュレータ内で走行させるロボットは,
図\ref{fig:raspicat}
の差動二輪型のロボットである.
差動二輪型のロボットは,
進行方向に対して前後に進むことができるが,
横に進むことができない.
そのため,当然ながらゴールの方向へ進むには
その方向を向く必要があり,
位置と方向の3次元で経路を探索することで効率化が計れる.

実験に用いた計算機は,CPUとしてRyzen 9 7945HX3D(16コア32スレッド),
DRAMとしてDDR5-4800 32GBが2枚(64GB)搭載されたものである.
価値反復ROSパッケージは,CPUですべて計算するため,
コアの数が重要であり,このコンピュータは,
現代の家庭用コンピュータの中でも,コアの数の多いCPUである.
DRAMは,地図の大きさに応じて決定した.

% スレッド数とゴール設定,比較対象は変更して実験をし直したい
% 大域計画だけにスレッドを割り振り
% 比較対象にA*経路にただ追従するものを追加して
価値反復ROSパッケージはマルチスレッド化されており,
環境の全域で$V$を計算する大域計画器と,
ロボットの周囲で$V$を計算する局所計画器に任意のスレッドの数
を割り当てることができる.
本稿の実験では,大域計画器に8,
局所計画器に6のスレッドを割り当てた.
A*は別のプロセスで動作する.
CPUから見ると,A*には1つのスレッドを割くこととなる.
また,シミュレーションにはGazeboを使用するため,
これにも計算機のリソースを使用している.

実験では,図\ref{fig:map}3つの目的地で比較を行った.
Goal1は...

\begin{table}[bt]
    \centering
    \caption{Configurations of the Map}
    \label{tab:map_info}
	\begin{small}
    \begin{tabular}{l|l}
        \hline
        map size & 294.6[m]$\times$199.95[m] \\
        cell resolution & 0.15[m]$\times$0.15[m] \\
        number of cells & 2,615,346 \\
        number of free cells & 165,076 \\
        the area of the free cells & 3714.98[m$^2$] \\
        \hline
    \end{tabular}
	\end{small}
\end{table}


% \begin{figure}[tbh]
%     \centering
%     \begin{minipage}{1 \linewidth}
%         \centering
%         \includegraphics[width=1 \linewidth]{figures/map.eps}
%     \end{minipage}
%     \caption{the Map for Navigation}
%     \label{fig:map_exp1}
% \end{figure}
% 
% 
% \begin{figure}[tbh]
%     \centering
%     \begin{minipage}{1 \linewidth}
%         \centering
%         \includegraphics[width=0.8 \linewidth]{figures/raspicat.eps}
%     \end{minipage}
%     \caption{Simulated Rasberry Pi Cat}
%     \label{fig:raspicat}
% \end{figure}


%\begin{figure}[hbt]
%    \centering
%  \includegraphics[width=0.8\linewidth]{figs/map.eps}
%  \caption{Map for experiment},
%  \label{fig:map}
%\end{figure}

\subsection{実験結果}
要追加

% 実機で実験したい.
\section{実機実験}



\chapter{結論}

よくなりました.


%%% 参考文献 %%%
% よほどのことが無い限りet al.は使わないことにしましょう
\bibliographystyle{jualpha}
\bibliography{./references}

% \newpage
% \printindex

\end{document}


