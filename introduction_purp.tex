\subsection{本研究の目的と構成}

\subsubsection{解決すべき課題}
以上の従来研究の調査から,価値反復法は移動ロボットの大域経路計画において
大域的最適性の保証,局所解の回避,不確実性への対処という優れた特性を持つことが確認された.
しかしながら,実環境での運用を考えた場合,以下の課題が依然として未解決,あるいは改善の余地がある.

\begin{enumerate}
    \item \textbf{動的環境へのリアルタイム追従性}: 環境の変化(ドアの開閉,荷物の移動など)に対し,
	全状態の価値関数を再計算するには時間がかかる.
	部分的な更新で整合性を保つ効率的なアルゴリズムが必要である.
    \item \textbf{コスト関数の設計困難性}: 安全性や社会的受容性(人への配慮)
	といった抽象的な指標を,どのような報酬関数として設計するかは自明ではない.
	逆強化学習(Inverse Reinforcement Learning)などのアプローチもあるが,計算負荷が高い.
    \item \textbf{3次元空間への拡張}: ドローンや不整地走行ロボットへの適用を考えた際,
	状態空間の次元爆発をどう抑えるかが課題となる.
\end{enumerate}


\section{研究目的}
価値反復ROSパッケージを用いたナビゲーションにおいて,
走り出しまでの時間を短縮することで
移動にかかる時間を短縮することを目的とする.


\section{論文の構成}
章では,問題設定と価値反復アルゴリズム,A*アルゴリズムについて述べ,
章では,提案手法,章では,実験について述べる.
章でまとめる.

