\section{研究目的}
以上から,価値反復ROSパッケージにおいて,
現代のコンピュータを用いて単純な経路を求める場合でも,
走り出すまでに時間がかかる問題があることが分かる.
ロボットが走り出すためには,
価値反復によって生成される$V$の勾配が,
ロボットの居る地点まで届く必要がある.

そこで,価値反復よりも計算量の少ないアルゴリズムで経路を算出し,
その経路を用いることで勾配を生成すると,
ロボットがより早いタイミングで走り始めることが期待できる.

よって,本研究の目的を,
「価値反復ROSパッケージを用いたナビゲーションにおいて,
走り出しまでの時間を短縮すること」,
とする.走り始めるまでの時間が短くなることで,
移動全体にかかる時間も短くすることができると考えられる.

$V$は,どのような初期値からも収束するため,
$V$の値を書き換えは,$V^*$に影響は与えないが,
計算中の$V$に対しては,影響を与えることができる.
この影響により全体の移動時間が伸びることがないようにしなければならない.



\section{論文の構成}
\ref{chap:introduction}章では,移動ロボット研究の背景,
従来研究,本研究の目的を述べた.

\ref{chap:problem}章では,移動ロボットの経路計画問題について述べる.

\ref{chap:method}章では,提案する手法について述べる.

\ref{chap:implement}章では,\ref{chap:method}章の手法の実装について述べる.

\ref{chap:experiment}章では,\ref{chap:method}章のアルゴリズムについて,
移動時間の短縮の効果を評価する.

\ref{chap:conclusion}章でまとめる.

