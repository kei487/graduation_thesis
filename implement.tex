\chapter{実装}

\section{A*の実装と価値反復への適用}
\subsection{A*の実装}
\subsubsection{2D A*の実装}
A*の探索空間は,$xy$平面であり,
次章の実験のための実装として,
ROS 2版のVIパッケージ\cite{[4]}に,
A*探索を実行するスレッドを1つ追加した.
この追加では,
\texttt{ike\_nav}\cite{[5]}
パッケージ内の\texttt{ike\_nav\_planner}をA*
探索のプログラムとして流用した.

\texttt{ike\_nav\_planner}は
マップと現在地と目的地に対してA*で計算された経路を出力する.
この経路上で,前節のように$V$の値を書き換えるコードを追加した.
ただし出力される経路は$XY$平面上のもので,$\theta$
方向の情報がない.そのため,$XY$平面上で同じ位置にある
離散状態にはすべて同じ値を式(\ref{eq:a-star_weight})
で代入し,あとは価値反復で$\theta$方向の$V$の値を計算させることとした.
この実装には改良の余地があり,
$XY\theta$空間内でのA*探索を用いるほうが,
より時間短縮の効果が得られるものと考えられる.

\subsubsection{3D A*の実装}
A*の具体的な実装については,
ike\_nav\cite{ikenav}パッケージ内の
A*探索ノード(ike\_nav\_planner)を3次元での探索に拡張し,
ROS 2版のVIパッケージ\cite{vipkg}に移植して利用した.


\subsection{価値反復とA*の併用}


\section{提案手法の実機への適用}
\subsection{コンピュータの選定}
実機に載せる都合上,サイズに気を使い,
選定する必要がある.
また,CPUを限界ギリギリまで長時間使用する都合上,
排熱も考慮する必要がある.

