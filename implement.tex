\chapter{実装}\label{chap:implement}

\section{A*の実装と価値反復への適用}
\subsection{A*の実装}
\subsubsection{2D A*の実装}
価値反復ROSパッケージに,A*探索を行うスレッドを
1つ追加した.
ROS 2のノードの形式で実装し,
実装には,\texttt{ike\_nav}\cite{[5]}
パッケージ内の\texttt{ike\_nav\_planner}をA*
探索のプログラムとして流用した.
\texttt{ike\_nav\_planner}は
マップと現在地と目的地に対してA*で計算された経路を出力する.
%もっと書く


\subsubsection{3D A*の実装}
そこで,$XY\theta$空間で探索を行うA*を実装した.
A*の具体的な実装については,
ike\_nav\cite{ikenav}パッケージ内の
A*探索ノード(ike\_nav\_planner)を3次元での探索に拡張し,
価値反復ROSパッケージに移植して利用した.
%もっと書く2 図もほしい
3D A*は,
価値反復と同様に,グリッドに加えて
$\theta$方向にも離散化し,グラフを構築する.
$XY$方向の遷移コストと$\theta$方向の遷移コストは単位が一致するよう
距離から時間に変更した.

ヒューリスティック関数$H$については,
\cite{nakamura2024}での平面上のユークリッド距離を計算するものから,
\begin{align}
	H(s) \triangleq W_1 \sqrt{  (x_\text{c} - x_\text{g})^2 + (y_\text{c} - y_\text{g})^2 }
	+ W_2 | \theta_\text{c} - \theta_\text{g} | \label{eq:3d-astar-heurisic}
\end{align}
に変更する.
ここで,
$(x_\text{c}, y_\text{c}, \theta_\text{c})$は各離散状態の中心の座標,
$(x_\text{g}, y_\text{g})$は目的地の中心地点の$xy$座標,
$\theta_\text{c} - \theta_\text{g}$は,$(x_\text{c}, y_\text{c}, \theta_\text{c})$
から見た$(x_\text{g}, y_\text{g})$の方角である.
$W_1, W_2$は定数である.
$W_1$と$W_2$の比は,ロボットが一定時間あたりに移動できる距離と方向転換できる角度の比で決まる.


\subsection{価値反復とA*の併用}
価値反復ROSパッケージ内に
A*で算出した経路上で,式(\ref{eq:purpose})
に従って,$V$の値を書き換えるコードを追加した.
ただし,2D A*は,出力される経路は$XY$平面上のもので,$\theta$
方向の情報がない.そのため,$XY$平面上で同じ位置にある
離散状態にはすべて同じ値を式(\ref{eq:a-star_weight})
で代入し,あとは価値反復で$\theta$方向の$V$の値を計算させることとした.
この実装には改良の余地があり,
$XY\theta$空間内でのA*探索を用いるほうが,
より時間短縮の効果が得られるものと考えられる.
%式もっかい出したり図入れたり

%余裕があれば書きたい
% \section{提案手法の実機への適用}
% \subsection{コンピュータの選定}
% 実機に載せる都合上,サイズに気を使い,
% 選定する必要がある.
% また,CPUを限界ギリギリまで長時間使用する都合上,
% 排熱も考慮する必要がある.
% 試したコンピュータを以下に示す.
