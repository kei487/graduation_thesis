\section{従来研究}
\subsection{移動ロボットにおける2種の経路計画器}

移動ロボットのナビゲーションにおける経路計画器は,
動的に目的地が決まる場合,
計算負荷と性能のバランスを保つため,
一般的に以下の2つの階層に分離して設計される.
目的地が固定されている場合は,
大域経路計画をアルゴリズムを使わずに決定することがある.

\begin{enumerate}
    \item \textbf{大域経路計画(Global Path Planning)}:
    環境全体の地図情報(事前地図)に基づき,スタートからゴールまでの大局的な最適ルートを生成する.
	更新頻度は比較的低く(数秒〜数分に1回,またはトポロジカルな構造の変更時),
	静的な障害物の回避を行う.
    \item \textbf{局所経路計画(Local Path Planning / Obstacle Avoidance)}:
    大域経路に追従しつつ,搭載されたセンサで検知した未知の障害物や動的障害物をリアルタイムに回避する
	ための制御入力を生成する.
	更新頻度は高く(10Hz〜100Hz),ロボットの運動学的制約や
	動力学的制約を考慮する.
	Dynamic Window Approach (DWA) やModel Predictive Control (MPC) などが代表的である.
\end{enumerate}


理由は後述するが,本研究は,このうち大域経路計画を取り扱う.
次章から大域経路計画の各手法について述べる.


\subsection{決定論的アプローチによる経路計画}
環境を,各要素を表すノードとそのノード同士の関係を表したエッジからなるグラフ構造として
モデル化することで,大域経路計画をグラフ探索問題に帰着できる.
素朴に大域経路計画を考えたとき,スタートからゴールへの一本のパスを見つける手法が考えられる.
グラフ探索問題は,与えられたグラフ内に,スタートとゴールのノードが設定され,
エッジをたどりノードを移動してゆき,最短の移動で,
ゴールのノードにたどり着く1通りのノードとエッジの列を求める問題である.


\subsubsection{グリッドベースの探索手法}
環境の地図を格子状(グリッド)に分割し,各セルをノード,
隣接セルへの移動をエッジとしてモデル化する手法である.
多くの場合,セルには,障害物があり通行不可能,障害物がなく通行可能,不明
の3種類があり,障害物がなく通行可能なセルからなるノードだけをたどる経路を算出することが求められる.
また,エッジは繋ぐノード間の距離を重みとして持つ.
多くの場合,1つのセルから隣接する8つのセルにエッジが繋がれており,
斜めに移動するエッジは,重みが$\sqrt{2}$倍になる.

\paragraph{Dijkstra法}
Edsger W. Dijkstraによって考案されたDijkstra法は,
非負の重み付きグラフにおける単一始点最短経路問題を解くアルゴリズムである\cite{Dijkstra1959}.
各エッジの重み(=距離)は,常に正であり,この重みを移動にかかるコストとして,
スタートのノード $n_s$ からゴールのノード $n_g$ までこのコストが最小になるような,
ノードの列(=経路)を算出する.
$n_s$ からあるノード $n$ まで,累加したコストを $g(n)$ とするとき,
$n_s$ から順に,移動可能な隣接するノードへの$g(n)$(=移動する距離)を計算し,
計算したノードの中から$g(n)$が最小のノードをコストが確定したノードとする.
新しく確定したノードから移動可能な隣接するノードの$g(n)$を再度計算し,
最小のものを選び,と探索範囲を広げていく.
常に $g(n)$ が最小となるノードから移動可能なノードからコストを計算する
ことで,数学的に最短経路が保証される.
しかし,探索が全方位に均等に広がるため,ゴールの方角情報利用されず,探索範囲が膨大になる欠点がある.

\paragraph{A*アルゴリズム}
A*(A-Star)アルゴリズムは,Dijkstra法にヒューリスティック関数 $h(n)$ 
を導入することで探索を効率化した手法である.\cite{Hart1968}
Dijkstra法の $g(n)$ に替わり用いる評価関数 $f(n)$ を以下のように定義する.
\begin{equation}
    f(n) = g(n) + h(n)
\end{equation}
ここで,$g(n)$ は,$n_s$ からノード $n$ までの実コスト,
$h(n)$ はノード $n$ から $n_g$ までの推定コストである.
$h(n)$ は,人の手によって設計され,
移動ロボットの場合は,$h(n)$ として現在のセルからゴールセルまでの
ユークリッド距離やマンハッタン距離が用いられる.
$h(n)$ が実際の最短コストを決して上回らない(許容的な,Admissible)場合,
A*アルゴリズムは最適解を保証しつつ,
Dijkstra法よりも少ない計算量で解に到達できることが多い.
現在でも最も広く使われている標準的なアルゴリズムである.

\begin{figure}
  \centering
  \includegraphics[width=0.8\linewidth]{example-image-16x9.pdf}
  \caption{Dijkstra法とA*アルゴリズムの探索範囲比較}
\end{figure}


\subsubsection{サンプリングベースの手法}
高次元の構成空間(Configuration Space)を持つロボット(多関節アームなど)や,
広大な環境においては,グリッドベースの手法は,グラフが大きくなり
探索にかかる計算量が多いものとなる.
これに対し,空間内の点をランダムにサンプリングしてグラフを構築する手法がある.

\paragraph{RRT (Rapidly-exploring Random Tree)}
LaValleによって提案されたRRTは,
ランダムにサンプリングされた点に向かってツリーを拡張していくことで,空間を急速に被覆する手法である\cite{LaValle1998}.
非ホロノミック拘束(自動車のような移動制約)を持つロボットの経路計画にも適用しやすい.
しかし,RRTによって生成される経路は最適性が保証されず,
ジグザグな無駄の多い経路になりがちである.
これを改良したRRT*(RRT-Star)\cite{Karaman2011}は,漸近的最適性を有するが,収束には時間を要する.

\paragraph{PRM (Probabilistic RoadMap)}
PRMは,学習フェーズで空間全体にランダムなノードを配置してロードマップ(グラフ)を構築し,
クエリフェーズでスタートとゴールをそのロードマップに接続して経路を探索する手法である\cite{Kavraki1996}.
多点間の移動を繰り返すようなタスクに適しているが,
狭い通路の通過が困難であるという問題が知られている.


\subsection{ポテンシャル法}
グラフ探索やサンプリング手法が,経路という点の集合を出力するのに対し,
環境全体に値を対応付けたポテンシャル関数を定義し,
ポテンシャル関数の勾配にしたがって移動する手法が存在する.
これらの手法は,自己位置推定のジャンプに対応しやすいことや,
大域経路計画と局所経路計画を同一の計算方法で計算できることといった利点がある.

\subsubsection{人工ポテンシャル法 (Artificial Potential Fields)}
Khatib\cite{khatib1986real}によって提案された人工ポテンシャル法は,
ロボットを仮想的な荷電粒子とみなし,
ゴールからの「引力」と障害物からの「斥力」を合成したポテンシャル場を構築する手法である.
ロボットはポテンシャルの勾配(Gradient)に従って最もエネルギーが低い方向へ移動するだけでよいため,
計算負荷が非常に軽く,リアルタイムな障害物回避に適している.
しかし,人工ポテンシャル法には,局所解(Local Minima)に陥ってしまうという問題がある.
U字型の障害物などに遭遇した場合,引力と斥力が釣り合ってしまい,
ゴールに到達する前に極小値で停止してしまう現象が発生する.

\subsubsection{ナビゲーション関数 (Navigation Functions)}
局所解の問題を解決するために,KoditschekとRimon\cite{koditschek1990robot}は,
ナビゲーション関数の概念を提唱した.
これは,幾何学的な構成空間において,ゴールのみを唯一の大域的最小点(Global Minimum)とし,
その他すべての停留点が不安定な鞍点(Saddle Point)となるように
設計された特殊なポテンシャル関数である.
ナビゲーション関数が構築できれば,その勾配に従うだけで,
ロボットはどのような初期位置からでも必ずゴールへ到達できることが数学的に保証される.

グリッドマップ上におけるナビゲーション関数の実装例として,
KonoligeのGradient Method\cite{konolige2000gradient}が挙げられる.
これは,Dijkstra法や波及法(Wavefront Propagation)を用いて
各グリッドセルからゴールまでの“真の距離コスト”を計算し,
これをポテンシャル関数とする.
こうして得られた場は局所解を持たず,大域的に最適な経路情報を内包している.


\subsection{価値反復法(Value Iteration)}
これらの決定論的な大域経路計画に対し,
本研究で取り扱う価値反復法は,
移動の不確実性や確率的な遷移を扱えるようにした
アルゴリズムである.
環境内のすべての状態(位置と向き)と値を対応付けた
状態価値関数(=一種のポテンシャル関数)と
行動を対応付けた方策(policy)を計算するアプローチである.
この手法は,環境の不確実性を確率的に扱うことが可能であり,
外乱に対してロバストなナビゲーションを実現する.
ナビゲーション関数やKonoligeの手法において計算される
「各地点からゴールまでの累積コスト」は,
数理的には決定論的な最適制御問題における価値関数(Value Function)と等価である.


\subsubsection{マルコフ決定過程(MDP)による定式化}
価値反復法では,環境をマルコフ決定過程(MDP)としてモデル化する.
MDPは,状態集合 $\mathcal{S}$,行動集合 $\mathcal{A}$,遷移確率 $\mathcal{P}$,報酬関数 $\mathcal{R}$ の4つ組で定義される\cite{中居2022}.
決定論的なグラフ探索とは異なり,MDPでは「行動 $a$ を行った結果,確率的に状態 $s'$ へ遷移する」
という不確実性をモデルに組み込むことができる.
これにより,例えば「スリップしやすい床」や「狭くて衝突リスクの高い通路」を,
単なる通行不可領域としてではなく,期待値として計算し,行動を判断できる.

\subsection{価値反復アルゴリズム}
最適方策(どのような行動をとるべきかのルール)を導出するために,
価値反復法では以下のベルマン最適方程式(Bellman Optimality Equation)を用いる.

\begin{equation}
    V^*(s) = \max_{a \in \mathcal{A}} \sum_{s' \in \mathcal{S}} \mathcal{P}_a(s, s') \left[ \mathcal{R}_a(s, s') + \gamma V^*(s') \right]
    \label{eq:bellman_fomula}
\end{equation}

この式を全状態に対して反復的に適用することで,
環境内のあらゆる場所からゴールへ向かうための最適な価値(ポテンシャル)が波及的に計算される.
すなわち,価値反復法によって得られる価値関数 $V^*(s)$ は,
確率的な状態遷移で計算したナビゲーション関数であると解釈できる.

この確率的な枠組みを用いることで,
ロボットが壁際を走行する際のリスクを考慮した経路計画が可能になることを示している\cite{thrun2005probabilistic}.
しかしながら,価値反復法は全状態空間を探索するため,広大な環境においては計算コストが甚大になるという課題が残されている.

ベルマン最適方程式 (\ref{eq:bellman_opt}) を用いて,
反復計算により $V^*(s)$ を求める手法が価値反復法(Value Iteration)である.
これはポテンシャル場に似ているが,ポテンシャル法が抱える局所解(Local Minima)の問題
(ゴール以外の窪みにハマって出られなくなる現象)が発生しないという強力な数学的保証がある.

\begin{figure}
  \centering
  \includegraphics[width=0.8\linewidth]{example-image-16x9.pdf}
  \caption{価値関数の伝播と収束プロセスの可視化}
\end{figure}


\subsubsection{不確実性の考慮と確率的MDP}
Thrunらの著書『Probabilistic Robotics』\cite{thrun2005}において,
センサノイズやアクチュエータ誤差を考慮したMDPベースのプランニングの重要性が説かれている.
例えば,狭い通路を通る際,A*アルゴリズムなどの幾何的な最短経路探索では,
壁ギリギリを通るルートを選択しがちである.
しかし,実ロボットには移動誤差があるため,壁に衝突するリスクが高い.
確率的な遷移モデル $\mathcal{P}$ を組み込んだ価値反復法を用いると,
壁際での移動は衝突して大きなペナルティを受ける確率がある行動として評価される.
その結果,多少遠回りであっても,壁から距離を取った安全な広い通路を選択するような,
リスク回避的な経路(Risk-Aware Path)が自然に生成される.
これは,安全性が最優先されるサービスロボットにおいて極めて重要な特性である.




\subsubsection{価値反復ROSパッケージ}
上田らは,将来的には,計算機の性能が向上し,この計算コストの問題が解決されるとして
現在の移動ロボットで使われるミドルウェアのROS上で実装した.


価値反復法の最大の欠点は,計算コストである.
状態空間 $\mathcal{S}$ のサイズに対して計算量が線形
(1回の反復あたり $O(|\mathcal{S}||\mathcal{A}|)$)に増加する.
広大な環境を高い解像度でグリッド化すると,状態数は数百万から数千万に達する.
