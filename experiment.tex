\chapter{実験}\label{chap:experiment}


\section{シミュレータ実験}

提案手法の走り出しまでの時間を短縮する効果の有無を
検証するために,シミュレータを用いて実験を行う.
シミュレータ環境では,動的な障害物が存在せず,
障害物回避による移動時間の増加を考慮する必要がない
理想的な環境で効果を検証できる.


\subsection{実験条件}
シミュレータ実験では,
千葉工業大学津田沼キャンパスで得られた
図\ref{fig:map}の地図を使用する.
紙面横方向が300m,縦方向が200mの6haの環境の地図で,
形式は,ROSのナビゲーションスタックで用いられる
占有格子地図である.
白色が障害物のない画素,
黒色が障害物のある画素,灰色が不明な画素を表す.
この地図のパラメータは表\ref{tab:map}の通りである.

価値反復ROSパッケージでは,地図のグリッドに加え,
$\theta$方向にも離散化を行い$mathbb{S}$を構成する.
$S$の諸元を表\ref{tab:vi_s}に示す.
要素数は1億に達し,移動可能なエリアだけでも1千万に達する.

実験で使用するシミュレータは,
図\ref{fig:map}から作成したシミュレータを使用する.
シミュレータには,ROSの標準的なシミュレータである
Gazeboを使用し,
シミュレータ内で用いる静的な障害物の作成には,
占有格子地図から作成するプログラム\cite{ikebe2020sim}
を使用した.

シミュレータ内で走行させるロボットは,
図\ref{fig:raspicat}
の差動二輪型のロボットである.
差動二輪型のロボットは,
進行方向に対して前後に進むことができるが,
横に進むことができない.
そのため,当然ながらゴールの方向へ進むには
その方向を向く必要があり,
位置と方向の3次元で経路を探索することで効率化が計れる.

実験に用いた計算機は,CPUとしてRyzen 9 7945HX3D(16コア32スレッド),
DRAMとしてDDR5-4800 32GBが2枚(64GB)搭載されたものである.
価値反復ROSパッケージは,CPUですべて計算するため,
コアの数が重要であり,このコンピュータは,
現代の家庭用コンピュータの中でも,コアの数の多いCPUである.
DRAMは,地図の大きさに応じて決定した.

% スレッド数とゴール設定,比較対象は変更して実験をし直したい
% 大域計画だけにスレッドを割り振り
% 比較対象にA*経路にただ追従するものを追加して
価値反復ROSパッケージはマルチスレッド化されており,
環境の全域で$V$を計算する大域計画器と,
ロボットの周囲で$V$を計算する局所計画器に任意のスレッドの数
を割り当てることができる.
本稿の実験では,大域計画器に8,
局所計画器に6のスレッドを割り当てた.
A*は別のプロセスで動作する.
CPUから見ると,A*には1つのスレッドを割くこととなる.
また,シミュレーションにはGazeboを使用するため,
これにも計算機のリソースを使用している.

実験では,図\ref{fig:map}3つの目的地で比較を行った.
Goal1は...

\begin{table}[bt]
    \centering
    \caption{Configurations of the Map}
    \label{tab:map_info}
	\begin{small}
    \begin{tabular}{l|l}
        \hline
        map size & 294.6[m]$\times$199.95[m] \\
        cell resolution & 0.15[m]$\times$0.15[m] \\
        number of cells & 2,615,346 \\
        number of free cells & 165,076 \\
        the area of the free cells & 3714.98[m$^2$] \\
        \hline
    \end{tabular}
	\end{small}
\end{table}


% \begin{figure}[tbh]
%     \centering
%     \begin{minipage}{1 \linewidth}
%         \centering
%         \includegraphics[width=1 \linewidth]{figures/map.eps}
%     \end{minipage}
%     \caption{the Map for Navigation}
%     \label{fig:map_exp1}
% \end{figure}
% 
% 
% \begin{figure}[tbh]
%     \centering
%     \begin{minipage}{1 \linewidth}
%         \centering
%         \includegraphics[width=0.8 \linewidth]{figures/raspicat.eps}
%     \end{minipage}
%     \caption{Simulated Rasberry Pi Cat}
%     \label{fig:raspicat}
% \end{figure}


%\begin{figure}[hbt]
%    \centering
%  \includegraphics[width=0.8\linewidth]{figs/map.eps}
%  \caption{Map for experiment},
%  \label{fig:map}
%\end{figure}

\subsection{実験結果}
要追加

% 実機で実験したい.
\section{実機実験}


