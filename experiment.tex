\chapter{実験}\label{chap:experiment}


\section{シミュレーター実験}
\subsection{実験条件}
%図\ref{fig:map}から作成した
シミュレーターで実験した.
千葉工業大学津田沼キャンパスで得られた
地図を用いたシミュレーションで,提案手法を評価する.
図?%\ref{fig:map_exp1}
に,この地図を示す.
紙面横方向が300m,縦方向が200mの6haの環境の地図で,
形式は,ROSのナビゲーションスタックで用いられる
占有格子地図である.
白色が障害物のない画素,
黒色が障害物のある画素,灰色が不明な画素を表す.
この地図のパラメータは表\ref{tab:map_exp1}の通りである.

シミュレータ内で走行させるロボットは,
図?% \ref{fig:raspicat}
に見られる差動二輪型のロボットである.

実験に用いた計算機は,CPUとしてRyzen 9 7945HX3D(16コア32スレッド),
DRAMとしてDDR5-4800 32GBが2枚(64GB)搭載されたものである.

VIパッケージはマルチスレッド化されており,
環境の全域で$V$を計算する大域計画器と,
ロボットの周囲で$V$を計算する局所計画器に任意のスレッドの数
を割り当てることができる.
本稿の実験では,大域計画器に8,
局所計画器に6のスレッドを割り当てた.
A*は別のプロセスで動作する.
CPUから見ると,A*には1つのスレッドを割くこととなる.
また,シミュレーションにはGazebo
(ROSの標準的なシミュレータ)を使用するので,
これにも計算機のリソースを使うこととなる.

\begin{table}[bt]
    \centering
    \caption{Configurations of the Map}
    \label{tab:map_info}
	\begin{small}
    \begin{tabular}{l|l}
        \hline
        map size & 294.6[m]$\times$199.95[m] \\
        cell resolution & 0.15[m]$\times$0.15[m] \\
        number of cells & 2,615,346 \\
        number of free cells & 165,076 \\
        the area of the free cells & 3714.98[m$^2$] \\
        \hline
    \end{tabular}
	\end{small}
\end{table}


% \begin{figure}[tbh]
%     \centering
%     \begin{minipage}{1 \linewidth}
%         \centering
%         \includegraphics[width=1 \linewidth]{figures/map.eps}
%     \end{minipage}
%     \caption{the Map for Navigation}
%     \label{fig:map_exp1}
% \end{figure}
% 
% 
% \begin{figure}[tbh]
%     \centering
%     \begin{minipage}{1 \linewidth}
%         \centering
%         \includegraphics[width=0.8 \linewidth]{figures/raspicat.eps}
%     \end{minipage}
%     \caption{Simulated Rasberry Pi Cat}
%     \label{fig:raspicat}
% \end{figure}


%\begin{figure}[hbt]
%    \centering
%  \includegraphics[width=0.8\linewidth]{figs/map.eps}
%  \caption{Map for experiment},
%  \label{fig:map}
%\end{figure}

\subsection{実験結果}

% 実機で実験したい.
\section{実機実験}


