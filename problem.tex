\chapter{移動ロボットの経路計画問題}\label{chap:purpose}

\section{扱う問題}
本研究では,移動ロボットを平面上で自律移動させる問題を扱う.
広く用いられている Nav2[?]%\cite{}] 
(ROS 2 の標準ナビゲーションパッケージ)が扱う問題と同様の
問題である.
ロボットは,行動を始めるタイミングで目的地の座標を与えられ,
障害物との衝突を避けながらできる限り短い時間で目的地まで
到達しなければならない.
ロボットが移動を行う空間を環境と言い,
図2%\ref{fig:map}
にその環境の例を示す.
環境には,世界座標系が 2 次元の直交座標系で設定されており,
ロボットは,位置$(x, y)$と,$x$軸となす角$\theta$を向きとして
持っており,これらをまとめてロボットの状態(位置と向き)$x = (x, y, \theta)$
3 変数で表現される.
目的地地点は図中の destination area のように $XY$
平面上の領域や,$xy\theta$空間内の領域として与えられる.
環境中には,その位置は既知である固定障害物と
移動障害物が存在するが,これらは VI パッケージの既存の機能で対処可能である.
しかし,本研究では移動障害物の回避は陽には扱わない.

\subsection{マルコフ決定過程(Markov Decision Process: MDP)による定式化}

MDPは,エージェントが環境と相互作用しながら学習・行動決定を行うための数理モデルであり,
以下の4つの要素の組 
$\langle \mathcal{S}, \mathcal{A}, \mathcal{P}, \mathcal{R} \rangle$ で定義される.

\begin{enumerate}
    \item \textbf{状態集合 $\mathcal{S}$ (State Space)}:
    ロボットが取り得るすべての状態の集合.2次元グリッドマップ上での経路計画の場合,
	各グリッドセル $(x, y)$ が一つの状態 $s \in \mathcal{S}$ に対応する.
	さらに,ロボットの方位 $\theta$ を含めて $(x, y, \theta)$ を状態とすることもある.

    \item \textbf{行動集合 $\mathcal{A}$ (Action Space)}:
    各状態でロボットが選択可能な行動の集合.
	グリッドマップ上では,隣接する8近傍(上下左右+斜め)への移動や,
	その場での停止などが行動 $a \in \mathcal{A}$ となる.

    \item \textbf{遷移確率 $\mathcal{P}_a(s, s')$ (Transition Probability)}:
    状態 $s$ で行動 $a$ を選択したときに,次の時刻に状態 $s'$ へ遷移する確率.
    \begin{equation}
        \mathcal{P}_a(s, s') = \Pr(S_{t+1}=s' \mid S_t=s, A_t=a)
    \end{equation}
    決定論的な環境(A*などが想定する世界)では,ある行動を行えば100\%意図した隣接セルへ移動する.
	しかし実環境では,タイヤのスリップや制御誤差により,意図したセルへ移動できない場合がある.
	MDPではこの不確実性を確率分布として明示的にモデル化できる.
	例えば,「前進」を選択しても,10\%の確率で「横滑り」する,といった表現が可能である.

    \item \textbf{報酬関数 $\mathcal{R}_a(s, s')$ (Reward Function)}:
    状態遷移に伴って得られる即時報酬(またはコスト).
	経路計画問題においては,通常「コストの最小化」または「負の報酬の最大化」として定式化される.
    例えば,ゴール状態に到達したときに大きな正の報酬を与え,
	障害物に衝突したときに大きな負の報酬(ペナルティ)を与える.
	また,移動にかかる時間やエネルギーを表現するため,
	各ステップごとにわずかな負の報酬(ステップコスト)を与える.
\end{enumerate}

\subsubsection{ベルマン方程式と価値関数}
MDPの目的は,各状態でどのような行動をとるべきかというルール,
すなわち「方策(Policy) $\pi: \mathcal{S} \to \mathcal{A}$」を見つけることである.
最適な方策 $\pi^*$ を見つけるために,「状態価値関数 $V^\pi(s)$」を導入する.
これは,ある状態 $s$ からスタートし,方策 $\pi$ に従って行動し続けたときに
得られる将来の報酬の総和(期待値)である.


割引率を $\gamma$ ($0 \le \gamma < 1$) とすると,価値関数は以下のように定義される.
\begin{equation}
    V^\pi(s) = \mathbb{E} \left[ \sum_{t=0}^{\infty} \gamma^t R_{t+1} \mid S_0 = s, \pi \right]
\end{equation}


最適な方策 $\pi^*$ に従ったときの価値関数を最適状態価値関数 $V^*(s)$ と呼ぶ.
$V^*(s)$ は,以下のベルマン最適方程式(Bellman Optimality Equation)を満たす.
\begin{equation}
    V^*(s) = \max_{a \in \mathcal{A}} \sum_{s' \in \mathcal{S}} 
	\mathcal{P}_a(s, s') \left[ \mathcal{R}_a(s, s') + \gamma V^*(s') \right]
    \label{eq:bellman_opt}
\end{equation}


この方程式は再帰的な構造をしており,
「ある状態の価値は,そこで最適な行動をとった際に期待される即時報酬と,
遷移先の状態の価値の割引和によって決まる」ことを意味している.



\section{価値反復アルゴリズム}
VIパッケージの価値反復で計算される式は,
ロボットがある位置・向き
$\boldsymbol{x}$にあるとき,
そこから目的地までのコストを計算した
状態価値関数
\begin{align}
	V: \mathcal{S} \rightarrow \mathbb{R}
\end{align}
である.ここで$\mathcal{S}$は,
$XY\theta$空間を格子状に離散化した
離散状態$s$の集合である.
また,コストというのは,時間と,
時間換算で与えるペナルティーを足した値である.
ペナルティーは悪路に相当する$s$など,ロボットが
入るとなんらかの問題のある状態に与えられる.

$V$は,価値反復の計算が進むと正解の値に近づいていく.
このとき,$V$の値は図\ref{fig:propose}(b)のように
目的地の周りから収束していき,
目的地から遠いところが最後に収束する.
この計算は探索ではないが,
「幅優先探索」に相当する処理となる.


$V$が収束すると,ロボットは$V$の値が小さくなるような行動を
選択し続けることで,最適な経路を選択して移動できるようになる.
また,$V$が完全に収束しなくても,
ロボットのいる$s$から目的地に向かって値が
単調減少していると,ロボットは目的地に向かうことができる.
このような状態を,本稿では「経路が見つかる」と表現する.


1章で述べたとおり,
価値反復は状態価値関数$V$を計算する.
VIパッケージの実装では,
$xy\theta$空間を格子状に離散化して
作った離散状態の集合$\mathcal{S}$の要素$s$
に対し,ひとつひとつコスト$V(s)$が計算される.
%つまり,
%$V: \mathcal{S} \rightarrow \mathbb{R}$
%という形式をとる.


$V(s)$には初期値として,$s$が目的地の領域
に含まれる場合に$0$,
そうでないときは無限大(実装上は目的地まで10万秒など,
非常に大きな値)が与えられる.
価値反復は終端状態以外の$V(s)$の値を,
繰り返し計算で実際のコストまで値を小さくしていく.


完全に収束した$V$は最適状態価値関数$V^*$と呼ばれ,
$V^*$からは,最適な行動を求めることができる.
ただし,ロボットが移動を開始するためには,$V^*$
を厳密に計算する必要はない.
現在地の周辺において,目的地に向かうための適切な
勾配が$V$にできると,
ロボットは目的地に向かう($V$のコストを減らす)
ように行動をとることができる.

しかしながら,この勾配の伝播は
A*などの探索手法が経路を探すよりも遅い.
価値反復は,探索の手法として見ると幅優先探索になっており,
$V$の勾配は目的地の周辺からできはじめ,
それがより遠い地点に伝播していく.
またこの伝搬の計算は,ロボットが通らないであろう
経路でも行われる.
そのため,目的地までの経路が存在すれば,
それが複雑に入り組んでいても
必ず見つけることはできるが,
ロボットは現在地に勾配が到達するまで,
長く待たされることになる.
価値反復アルゴリズムは,ベルマン方程式の形式に乗るように
定式化することで,最適な状態価値関数を得られるアルゴリズムである.



\subsection{価値反復アルゴリズム}
ベルマン最適方程式 (\ref{eq:bellman_opt}) を用いて,
反復計算により $V^*(s)$ を求める手法が価値反復法(Value Iteration)である.
アルゴリズムの手順は以下の通りである.

\begin{enumerate}
    \item \textbf{初期化}: すべての状態 $s$ について,$V(s)$ を任意の値(通常は0)に初期化する.
    \item \textbf{反復更新}: 以下の更新式を,すべての状態 $s$ に対して適用する.
    \begin{equation}
        V_{k+1}(s) \leftarrow \max_{a \in \mathcal{A}} \sum_{s' \in \mathcal{S}} 
	\mathcal{P}_a(s, s') \left[ \mathcal{R}_a(s, s') + \gamma V_k(s') \right]
    \end{equation}
    ここで $k$ は反復回数を表す.
    \item \textbf{収束判定}: 全状態における価値関数の更新量 $|V_{k+1}(s) - V_k(s)|$ の最大値が,
	事前に定めた閾値 $\epsilon$ 未満になれば停止する.
    \item \textbf{方策抽出}: 収束した価値関数 $V^*(s)$ を用い,
	各状態で価値を最大化する行動を選択する貪欲方策(Greedy Policy)を得る.
    \begin{equation}
        \pi^*(s) = \operatorname*{argmax}_{a \in \mathcal{A}} \sum_{s' \in \mathcal{S}} 
	\mathcal{P}_a(s, s') \left[ \mathcal{R}_a(s, s') + \gamma V^*(s') \right]
    \end{equation}
\end{enumerate}

この計算により,環境内のあらゆる場所からゴールへ向かうための最適な「勾配」が得られる.
これはポテンシャル場に似ているが,ポテンシャル法が抱える「局所解(Local Minima)」の問題
(ゴール以外の窪みにハマって出られなくなる現象)が発生しないという強力な数学的保証がある.
なぜなら,ベルマン方程式による更新は,大域的な最適性を伝播させる処理だからである.

% \section{経路探索アルゴリズム}
% 
% \subsection{Dijsktra法}
% よく知られたグラフ探索アルゴリズムである.
% 
% \subsection{A*アルゴリズム}
% Dijkstra法を発展させたものであり,ゴールまでのコストを推定する
% ヒューリスティック関数を追加したものである.
