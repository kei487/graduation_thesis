\chapter{移動ロボットの経路計画問題}\label{chap:purpose}

\section{扱う問題}
本研究では,移動ロボットを平面上で自律移動させる問題を扱う.
広く用いられている Nav2[?]%\cite{}] 
(ROS 2 の標準ナビゲーションパッケージ)が扱う問題と同様の
問題である.
ロボットは,行動を始めるタイミングで目的地の座標を与えられ,
障害物との衝突を避けながらできる限り短い時間で目的地まで
到達しなければならない.
ロボットが移動を行う空間を環境と言い,
図2%\ref{fig:map}
にその環境の例を示す.
環境には,世界座標系が 2 次元の直交座標系で設定されており,
ロボットは,位置$(x, y)$と,$x$軸となす角$\theta$を向きとして
持っており,これらをまとめてロボットの状態(位置と向き)$x = (x, y, \theta)$
3 変数で表現される.
目的地地点は図中の destination area のように $XY$
平面上の領域や,$xy\theta$空間内の領域として与えられる.
環境中には,その位置は既知である固定障害物と
移動障害物が存在するが,これらは VI パッケージの既存の機能で対処可能である.
しかし,本研究では移動障害物の回避は陽には扱わない.

\section{価値反復}
価値反復アルゴリズムは,ベルマン方程式の形式に乗るように
定式化することで,最適な状態価値関数を得られるアルゴリズムである.

\section{経路探索アルゴリズム}

\subsection{Dijsktra法}
よく知られたグラフ探索アルゴリズムである.

\subsection{A*アルゴリズム}
Dijkstra法を発展させたものであり,ゴールまでのコストを推定する
ヒューリスティック関数を追加したものである.
