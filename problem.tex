\chapter{移動ロボットの経路計画問題}\label{chap:purpose}

\section{扱う問題}
本研究では,移動ロボットを平面上で自律移動させる問題を扱う.
広く用いられている Nav2[?]%\cite{}] 
(ROS 2 の標準ナビゲーションパッケージ)が扱う問題と同様の
問題である.
ロボットは,行動を始めるタイミングで目的地の座標を与えられ,
障害物との衝突を避けながらできる限り短い時間で目的地まで
到達しなければならない.
ロボットが移動を行う空間を環境と言い,
図2%\ref{fig:map}
にその環境の例を示す.
環境には,世界座標系が 2 次元の直交座標系で設定されており,
ロボットは,位置$(x, y)$と,$x$軸となす角$\theta$を向きとして
持っており,これらをまとめてロボットの状態(位置と向き)$x = (x, y, \theta)$
3 変数で表現される.
目的地地点は図中の destination area のように $XY$
平面上の領域や,$xy\theta$空間内の領域として与えられる.
環境中には,その位置は既知である固定障害物と
移動障害物が存在するが,これらは VI パッケージの既存の機能で対処可能である.
しかし,本研究では移動障害物の回避は陽には扱わない.

\section{価値反復}
VIパッケージの価値反復で計算される式は,
ロボットがある位置・向き
$\boldsymbol{x}$にあるとき,
そこから目的地までのコストを計算した
状態価値関数
\begin{align}
	V: \mathcal{S} \rightarrow \mathbb{R}
\end{align}
である.ここで$\mathcal{S}$は,
$XY\theta$空間を格子状に離散化した
離散状態$s$の集合である.
また,コストというのは,時間と,
時間換算で与えるペナルティーを足した値である.
ペナルティーは悪路に相当する$s$など,ロボットが
入るとなんらかの問題のある状態に与えられる.

$V$は,価値反復の計算が進むと正解の値に近づいていく.
このとき,$V$の値は図\ref{fig:propose}(b)のように
目的地の周りから収束していき,
目的地から遠いところが最後に収束する.
この計算は探索ではないが,
「幅優先探索」に相当する処理となる.


$V$が収束すると,ロボットは$V$の値が小さくなるような行動を
選択し続けることで,最適な経路を選択して移動できるようになる.
また,$V$が完全に収束しなくても,
ロボットのいる$s$から目的地に向かって値が
単調減少していると,ロボットは目的地に向かうことができる.
このような状態を,本稿では「経路が見つかる」と表現する.


1章で述べたとおり,
価値反復は状態価値関数$V$を計算する.
VIパッケージの実装では,
$xy\theta$空間を格子状に離散化して
作った離散状態の集合$\mathcal{S}$の要素$s$
に対し,ひとつひとつコスト$V(s)$が計算される.
%つまり,
%$V: \mathcal{S} \rightarrow \mathbb{R}$
%という形式をとる.


$V(s)$には初期値として,$s$が目的地の領域
に含まれる場合に$0$,
そうでないときは無限大(実装上は目的地まで10万秒など,
非常に大きな値)が与えられる.
価値反復は終端状態以外の$V(s)$の値を,
繰り返し計算で実際のコストまで値を小さくしていく.


完全に収束した$V$は最適状態価値関数$V^*$と呼ばれ,
$V^*$からは,最適な行動を求めることができる.
ただし,ロボットが移動を開始するためには,$V^*$
を厳密に計算する必要はない.
現在地の周辺において,目的地に向かうための適切な
勾配が$V$にできると,
ロボットは目的地に向かう($V$のコストを減らす)
ように行動をとることができる.

しかしながら,この勾配の伝播は
A*などの探索手法が経路を探すよりも遅い.
価値反復は,探索の手法として見ると幅優先探索になっており,
$V$の勾配は目的地の周辺からできはじめ,
それがより遠い地点に伝播していく.
またこの伝搬の計算は,ロボットが通らないであろう
経路でも行われる.
そのため,目的地までの経路が存在すれば,
それが複雑に入り組んでいても
必ず見つけることはできるが,
ロボットは現在地に勾配が到達するまで,
長く待たされることになる.
価値反復アルゴリズムは,ベルマン方程式の形式に乗るように
定式化することで,最適な状態価値関数を得られるアルゴリズムである.

\section{経路探索アルゴリズム}

\subsection{Dijsktra法}
よく知られたグラフ探索アルゴリズムである.

\subsection{A*アルゴリズム}
Dijkstra法を発展させたものであり,ゴールまでのコストを推定する
ヒューリスティック関数を追加したものである.
