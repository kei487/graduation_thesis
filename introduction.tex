\chapter{序論}

\section{研究背景}
自律移動ロボットの経路計画技術は,
移動ロボットを実現する技術の1つのパーツである.
移動ロボットは,
従来,人が行っていた作業を代替することによって,
少子高齢化による労働人口の減少問題の解決に寄与すると考えられ,
研究のみならず実社会への導入が進んでいる.
具体例には,図1%\ref{fig:robots_in_real_world}
に示すように,
ビルや商業施設の警備や店舗や家屋の清掃,運搬,配膳,農業といった分野で
導入されている.
導入される事例があり,重要度が上がっている.


これらのロボットは,自律移動を行うことによって,
作業の代替をなしている.
自律移動は,人の操作を必要とせず,
目的地まで安全に,移動することが求められる.


ロボットが安全に移動するためには,建造物やロボットが乗り越えられない段差といった
短い時間で位置の変化しない障害物と,人間や停車している車,看板といった短い時間で位置の
変化する障害物の両方を回避ないしは停止し,ぶつからないことが求められる.
自律移動に用いられる技術は,大別して自己位置推定,経路計画の2つの要素技術から
構成されている.
自己位置推定は,周囲の環境からには,MCL(Monte Carlo Localization)\cite{fox1999}
が用いられる.


現在,経路計画にはよくA*アルゴリズムが用いられ,
経路追従と障害物回避にはDWA(Dynamic Window Approch)が用いられる.
このように異なるアルゴリズムを用いることにより,計算が簡単になる反面
競合する可能性が残っている.
そのため,経路計画と障害物回避を同時に行う手法として,
ポテンシャル法がある.
でも局所最適に陥る問題があり,価値反復はそれを解決する.



\section{従来研究}
上田らは価値反復ROSパッケージを開発した.


\section{研究目的}
価値反復ROSパッケージを用いたナビゲーションにおいて,
走り出しまでの時間を短縮することで
移動にかかる時間を短縮することを目的とする.


\section{論文の構成}
章では,問題設定と価値反復アルゴリズム,A*アルゴリズムについて述べ,
章では,提案手法,章では,実験について述べる.
章でまとめる.

