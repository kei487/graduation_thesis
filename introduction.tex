\chapter{序論}

\section{研究背景}
経路計画技術は,現代社会において必要性が見込まれる
自律移動システムの要素技術である.
少子高齢化によって労働人口が減少している問題に対して,
自律移動システムを持つロボットが,
解決に寄与すると考えられている.
ビルや商業施設の警備や,店舗や家屋の清掃,配膳,農業といった,
従来,人が行っていた作業は,移動を基礎とした作業である.
そのため,自律移動システムは,
これらの作業をロボットが代替する際に,要求される.


自律移動システムは,移動するロボットが,
人の操作を必要とせずに,安全に移動を行うシステムである.
ロボットが安全に移動するためには,
障害物を回避ないしは停止し,ぶつからないことが求められる.
障害物は,建造物やロボットが乗り越えられない段差といった
短い時間で位置の変化しない固定障害物と,
人間や停車している車,看板といった短い時間で位置の変化する移動障害物
の2種類に分けられる.

自律移動システムには,
大別して自己位置推定,経路計画の2つの要素技術がある.


自己位置推定は,ロボット自身が
環境からセンサを通じて得られる情報から
環境内の自身の位置を推定する技術である.
センサ情報から自身の位置を推定
する技術である.
には,MCL(Monte Carlo Localization)\cite{fox1999}
が用いられる.






現在,経路計画にはよくA*アルゴリズムが用いられ,
経路追従と障害物回避にはDWA(Dynamic Window Approch)が用いられる.
このように異なるアルゴリズムを用いることにより,計算が簡単になる反面
競合する可能性が残っている.
そのため,経路計画と障害物回避を同時に行う手法として,
ポテンシャル法がある.
でも局所最適に陥る問題があり,価値反復はそれを解決する.



\section{従来研究}
上田らは価値反復ROSパッケージを開発した.


\section{研究目的}
価値反復ROSパッケージを用いたナビゲーションにおいて,
走り出しまでの時間を短縮することで
移動にかかる時間を短縮することを目的とする.


\section{論文の構成}
章では,問題設定と価値反復アルゴリズム,A*アルゴリズムについて述べ,
章では,提案手法,章では,実験について述べる.
章でまとめる.

