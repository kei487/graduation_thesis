\chapter{序論}

\section{研究背景}
\subsection{ロボティクスに対する社会的要請}

\subsubsection{少子高齢化と労働生産性の課題}
21世紀に入り,先進諸国において少子高齢化に伴う生産年齢人口(15〜64歳の人口)
の減少が深刻な社会問題となっている.
特に日本においては,国立社会保障・人口問題研究所の推計によると,
生産年齢人口は1995年をピークに減少傾向にあり,
2070年には約4,500万人(2020年比で約4割減)まで落ち込むことが予測されている\cite{jfpe2023}.
この人口構造の変化は,経済成長の鈍化のみならず,
社会インフラの維持そのものを脅かす要因となっている.

物流業界においては,物流クライシスと呼ばれる状況が顕在化している.
電子商取引(E-commerce)の爆発的な普及により,小口配送の需要が急増している.
国土交通省の調査によれば,宅配便取扱個数は年間50億個(2023年度)を超え,
過去10年間で約1.3倍に増加している\cite{sthd2024}
トラックドライバーや倉庫内作業員の不足は慢性化しており,
労働環境の悪化と配送網の維持困難が懸念されている.
また,製造業の現場においても,熟練工の引退に伴う技術継承の問題や,
単純搬送作業への人員配置の困難さが指摘されている.

さらに,医療・介護の分野では,高齢者人口の増加に対し,
介護従事者の数は圧倒的に不足している.
厚生労働省の推計では,2040年度には約272万人の介護職員が必要とされているが,
現状のままでは数十万人規模の供給不足に陥る可能性が指摘されている\cite{nbnp2024}.
病院内での検体搬送,リネン類の回収,あるいは介護施設における見守りや配膳など,
定型的な業務の負担軽減は,強い需要がある.


\subsubsection{ロボットによる業務自動化と限界}
こうした労働力不足を補う手段として,
工場内物流においては,無人搬送車(Automated Guided Vehicle: AGV)の
導入が進められてきた.
AGVは,床面に敷設された磁気テープや反射テープ,
あるいは二次元コードといった物理的なガイドをセンサで読み取りながら走行する
ロボットである.
これにより,従来人の行っていた台車を押すような運搬作業をロボットが代替する
ことが可能になった.
AGVは環境が固定されており,かつタスクが定型的である場合に高い効率と
信頼性を発揮する.

一方で,AGVの導入と運用には,
物理的なガイドを必要とするという制約から
以下のような構造的な限界が存在すると指摘されている\cite{fragapane2021}.
これらの限界は,人や有人フォークリフトが頻繁に行き交う物流倉庫や,
一般の人々が存在する病院・商業施設といった
動的な障害物の多い環境へのロボット導入を阻む大きな障壁となっていた.

\begin{enumerate}
    \item \textbf{インフラ敷設コスト}: AGVに走行させたい経路のすべてにガイドを設置する必要があり,
	導入時の工事コストや期間が甚大である.
    \item \textbf{レイアウト変更の柔軟性欠如}: 製造ラインや倉庫のレイアウトを変更する際,
	ガイドの敷設し直しが必要となり,多大なコストとダウンタイムが発生する.
    \item \textbf{動的環境への非適応}: 想定された経路上に障害物が存在した場合,
	AGVはその場で停止することしかできず,回避して目的地へ向かうことができない.
\end{enumerate}


\subsubsection{AGVから自律移動ロボット(Autonomous Mobile Robot: AMR)へ}
AGVの課題を克服するため研究,開発が進められてきたのが,
自律移動ロボット(AMR)である.
AMRは,LiDAR(Light Detection and Ranging)やカメラといった外界センサ
を通じて周囲の環境の情報を得る.
このセンサ情報と事前に作成した環境地図を照らし合わせることで
地図内の自身の位置を推定し,ガイドを必要とせず走行する.

AMRは,AGVと違い,経路上に障害物を検知すると,回避経路を生成し,
その経路を追従することでタスクを継続することが可能である.
これは,AMRが走行する経路は,
物理的なガイドによる経路ではなく,ソフトウェア上の経路を走行するために,
動的に経路を変更することが可能であるため可能である.

また,物理的なガイドを必要としないため,導入時の工事が不要であり,
ソフトウェア上の設定変更のみで走行エリアや経路を変更できる.
この特性により,AMRは従来AGVが導入困難であった動的な環境への適用が進んでおり,
Society 5.0の中核を担う技術として期待されている\cite{stbc2016}.

\begin{figure}
  \centering
  \includegraphics[width=0.8\linewidth]{example-image-16x9.pdf}
  \caption{AGVとAMRの比較図}
\end{figure}


\subsubsection{自律移動ロボットの屋外への適用}
工場や倉庫といった屋内環境で培われたAMRの技術は,近年,より複雑かつ広範な屋外環境へとその適用範囲を拡大している.
特に,労働力不足が深刻な物流や農業分野において,屋外対応型AMRの実用化が急速に進展している.

物流分野では,配送拠点からエンドユーザーへの最終区間である
ラストワンマイルの配送コスト削減が最大の課題となっている.
人の代わりに荷物の配送を行う自律移動ロボットは,従来のトラック配送と比較して,
配送時間の短縮と環境負荷の低減を実現する有効な手段として位置付けられている\cite{alverhed2024}.
日本国内においても,2023年4月に施行された改正道路交通法により,
自律移動ロボットが遠隔操作型小型車として定義され,届出制による歩道走行が可能となった\cite{npa2023}.
これにより,パナソニックや楽天といった企業が住宅街での配送実証を行っており,
社会実装が進みつつある.

また,農業分野では,農林水産省がスマート農業を推進しており\cite{satp2024},
北海道大学の野口らによる先駆的な研究\cite{noguchi2011}を基盤として,
クボタやヤンマーなどの農機メーカーが有人監視下での自動走行(レベル2)および無人自動走行(レベル3)に対応した
ロボットトラクターを市場投入している\cite{maff2023}.
同様に,建設現場や鉱山といった過酷な環境においても,資材搬送や巡回監視を行うAMRの導入が進められている.

\begin{figure}
  \centering
  \includegraphics[width=0.8\linewidth]{example-image-16x9.pdf}
  \caption{楽天とかの自動配送ロボットの写真を入れたい}
\end{figure}

しかしながら,屋内環境と比較して,
屋外環境はロボットにとってタスクの遂行が困難となる要素が多い.
第一に,
ロボットの走る路面のロボットへ与える影響がある.
多くの移動ロボットは車輪型\cite{hara2025}であり,
平らで傾きのない屋内では,路面からロボットへ与える影響は小さい.
一方,屋外では,路面に段差や凹凸があり,
ロボットの姿勢の急激な変化や振動によるセンサノイズといった影響がある.
第二に,
天候や季節による路面状況の変化や,
時間変化による照明条件の変化がある.
雨,泥,雪,あるいは落ち葉などは,
風景の見た目を大きく変化させる.
また,直射日光による白飛びや逆光,夜間の低照度,
朝日,夕日による色温度の変化
といった光環境の変動は,
視覚情報の大きな変化を伴う.

これらの環境要因から
ロボットがどのように周囲を認識し,どのように自身の位置を知り,どのように経路を引くか
という課題が発生する.
次項では,
自律移動を実現するために現代のAMRがどのような技術要素によって構成されているか,
どのような課題が存在するか,
そのシステム概要について述べる.


\subsection{自律移動ロボットの技術構成}

\subsubsection{センシング技術}
ロボットが周囲の環境から情報を得るための技術である.

\begin{itemize}
    \item \textbf{LiDAR}: レーザー光を照射し,反射光が戻ってくるまでの時間(Time of Flight)
	や位相差から距離を計測する.
	北陽に代表される
	2次元平面をスキャンする2D LiDARが主流であったが,
	近年では,HesaiやLivoxといった中国メーカーによる,
	3次元点群を取得可能な3D LiDARの低価格化が進んでいる.
    \item \textbf{カメラ}: RGB画像に加え,深度情報を取得できるRGB-Dカメラや
	ステレオカメラが利用される.
	Visual SLAMや物体認識との親和性が高い.
    \item \textbf{オドメトリ(Odometry)}: 車輪の回転数(エンコーダ値)や
	IMU(慣性計測装置)のデータから,ロボットの相対的な移動量を推定する.
	短期的には高精度だが,累積誤差が生じるため単独では長距離移動に適さない.
    \item \textbf{GNSS(Global Navigation Satellite System)/GPS(Global Position System)}: 
    GPSやQZSS(みちびき)を含む衛星測位システムの総称.
    屋外環境においては絶対位置を取得する主要な手段となる.
    搬送波位相を用いるRTK-GNSSにより,数センチメートルの精度での測位が可能となり,
    農業機械や自動配送ロボットで広く採用されている.
    \item \textbf{無線通信 (WiFi/5G)}: アクセスポイントからの信号強度(RSSI)や
    電波の到達時間(RTT)を用いた測位に利用される.
\end{itemize}

\subsubsection{自己位置推定(Localization)技術}
ロボット自身が今どこにいるかを計算する技術は,
自律移動にとって欠かせない技術である.
オドメトリの累積誤差を補正し,地図座標系上での絶対位置を特定するために,
様々な手法が開発されてきた.


\paragraph{確率的ロボティクスの枠組み}
1990年代以降,Thrunらによって提唱された
確率的ロボティクス(Probabilistic Robotics)\cite{thrun2005}は,
センサのノイズや環境の不確実性を確率分布として扱うことで,ロバストな自己位置推定を可能にした.
代表的な手法として,(拡張)カルマンフィルタと
パーティクルフィルタ(Particle Filter)がある.
パーティクルフィルタの実装方法として広く使われているのが,
モンテカルロ位置推定(Monte Carlo Localization: MCL)\cite{Fox1999}
である.
これは,ロボットの存在確率分布を多数の粒子(パーティクル)で表現し,
センサ観測と動作モデルに基づいて粒子の重みと位置を更新する手法である.
MCLは,ロバスト性において他の手法より優れており,
現在のAMRのデファクトスタンダードとなっている.

%\dummyfig{パーティクルフィルタによる位置推定の概念図}{4cm}

\paragraph{SLAM (Simultaneous Localization and Mapping)}
地図を持たない未知の環境においては,自己位置推定と同時に環境地図の作成を行うSLAM技術が用いられる.
\begin{itemize}
    \item \textbf{LiDAR SLAM}: GMappingやCartographerなど,
	レーザーレンジファインダを用いたグリッドマップ生成手法.
  GLIMに代表されるGraph-based SLAMと呼ばれる手法は,
  ロボットの姿勢(ノード)と観測制約(エッジ)をグラフ構造として表現し,
  非線形最小二乗法によって全体最適化を行うもの.
    \item \textbf{Visual SLAM}: ORB-SLAMなど,カメラ画像の特徴点を用いて
	疎な地図(Feature Map)を作成する.テクスチャの豊富な環境に強い.
\end{itemize}


\paragraph{経路計画(Path Planning)技術}
自己位置推定技術の成熟により,
ロボットは環境内での自身の位置を推定できるようになった.
また,SLAM技術により,障害物の配置や通行可能な領域を表す
環境の地図を作成することも可能となった.


ロボットが実際にタスクを遂行するためには,自己位置推定だけでなく,
現在地(Start)から目的地(Goal)まで,
障害物を回避しつつ,かつ効率的(最短時間,最小エネルギーなど)
に到達するための軌道を決定しなければならない.


経路計画は,自己位置推定の結果と環境地図,目的地を入力とし,
ロボットのアクチュエータ(モータ)への制御指令を出力とする,
自己位置推定がいかに正確なものであっても,
経路計画が不適切であれば,ロボットは遠回りをするか,
狭い通路で立ち往生するか,最悪の場合は動的障害物と衝突する危険性がある.


n章で述べたように,社会実装が進むにつれて,ロボットが稼働する環境はより複雑化している.
静的な障害物だけでなく,人や他のロボットが移動する動的環境や,
路面に存在する微小な障害物によるノイズや,環境の変化によるセンサ計測の不確実性
の存在するような環境が挙げられる.
これらの要因を考慮し,安全かつ最適な経路をリアルタイムに導出することは,
自律移動ロボットの社会実装を進めるにあたって,重要な研究課題である.
次節では,この経路計画に関する従来研究を概観し,
本研究で扱う価値反復法の位置付けを明らかにする.
% ↑できるか??


\section{従来研究}
\subsection{移動ロボットにおける2種の経路計画器}

移動ロボットのナビゲーションにおける経路計画器は,
動的に目的地が決まる場合,
計算負荷と性能のバランスを保つため,
一般的に以下の2つの階層に分離して設計される.
目的地が固定されている場合は,
大域経路計画をアルゴリズムを使わずに決定することがある.

\begin{enumerate}
    \item \textbf{大域経路計画(Global Path Planning)}:
    環境全体の地図情報(事前地図)に基づき,スタートからゴールまでの大局的な最適ルートを生成する.
	更新頻度は比較的低く(数秒〜数分に1回,またはトポロジカルな構造の変更時),
	静的な障害物の回避を行う.
    \item \textbf{局所経路計画(Local Path Planning / Obstacle Avoidance)}:
    大域経路に追従しつつ,搭載されたセンサで検知した未知の障害物や動的障害物をリアルタイムに回避する
	ための制御入力を生成する.
	更新頻度は高く(10Hz〜100Hz),ロボットの運動学的制約(Kinematics)や
	動力学的制約(Dynamics)を考慮する.
	Dynamic Window Approach (DWA) やModel Predictive Control (MPC) などが代表的である.
\end{enumerate}


理由は後述するが,本研究は,このうち大域経路計画を取り扱う.
次章から大域経路計画の各手法について述べる.


\subsection{決定論的アプローチによる経路計画}
環境を,各要素を表すノードとそのノード同士の関係を表したエッジからなるグラフ構造として
モデル化することで,大域経路計画をグラフ探索問題に帰着できる.
素朴に大域経路計画を考えたとき,スタートからゴールへの一本のパスを見つける手法が考えられる.
グラフ探索問題は,与えられたグラフ内に,スタートとゴールのノードが設定され,
エッジをたどりノードを移動してゆき,最短の移動で,
ゴールのノードにたどり着く1通りのノードとエッジの列を求める問題である.


\subsubsection{グリッドベースの探索手法}
環境の地図を格子状(グリッド)に分割し,各セルをノード,
隣接セルへの移動をエッジとみなす手法である.
多くの場合,セルには,障害物があり通行不可能,障害物がなく通行可能,不明
の3つのモードがあり,障害物がなく通行可能なセルだけをたどる経路を算出することが求められる.

\paragraph{Dijkstra法}
Edsger W. Dijkstraによって考案されたDijkstra法は,
非負の重み付きグラフにおける単一始点最短経路問題を解くアルゴリズムである.
スタートノードから順に,隣接ノードへの移動コストを累積し,
確定したノードから探索範囲を広げていく(幅優先探索のコスト付き版と言える).
あるノード $n$ までの最小コストを $g(n)$ とするとき,
常に $g(n)$ が最小となるノードを展開することで,数学的に最短経路が保証される.
しかし,探索が全方位に均等に広がるため,ゴールの方角情報利用されず,探索範囲が膨大になる欠点がある.

\paragraph{A*アルゴリズム}
A*(A-Star)アルゴリズムは,Dijkstra法にヒューリスティック関数 $h(n)$ 
を導入することで探索を効率化した手法である.Hartらによって提案された.
評価関数 $f(n)$ を以下のように定義する.
\begin{equation}
    f(n) = g(n) + h(n)
\end{equation}
ここで,$g(n)$ はスタートからノード $n$ までの実コスト,
$h(n)$ はノード $n$ からゴールまでの推定コスト(ヒューリスティック)である.
移動ロボットの場合,$h(n)$ として現在のセルからゴールセルまでの
ユークリッド距離やマンハッタン距離が用いられる.
$h(n)$ が実際の最短コストを決して上回らない(許容的な,Admissible)場合,
A*アルゴリズムは最適解を保証しつつ,Dijkstra法よりも少ないノード展開数で解に到達できる.
現在でも最も広く使われている標準的なアルゴリズムである.

%\dummyfig{
\begin{figure}
  \centering
  \includegraphics[width=0.8\linewidth]{example-image-16x9.pdf}
  \caption{Dijkstra法とA*アルゴリズムの探索範囲比較}
\end{figure}


\subsubsection{サンプリングベースの手法}
高次元の構成空間(Configuration Space: C-Space)を持つロボット(多関節アームなど)や,
広大な環境においては,グリッドベースの手法は次元の呪いにより,グラフが極めて大きいものになり
探索にかかる計算量が多いものとなる.
これに対し,空間内の点をランダムにサンプリングしてグラフを構築する手法が提案されている.

\paragraph{RRT (Rapidly-exploring Random Tree)}
LaValleによって提案されたRRTは,
ランダムにサンプリングされた点に向かってツリーを拡張していくことで,空間を急速に被覆する手法である.
非ホロノミック拘束(自動車のような移動制約)を持つロボットの経路計画にも適用しやすい.
しかし,RRTによって生成される経路は最適性が保証されず,
ジグザグな無駄の多い経路になりがちである.
これを改良したRRT*(RRT-Star)は,漸近的最適性を有するが,収束には時間を要する.

\paragraph{PRM (Probabilistic RoadMap)}
PRMは,学習フェーズで空間全体にランダムなノードを配置してロードマップ(グラフ)を構築し,
クエリフェーズでスタートとゴールをそのロードマップに接続して経路を探索する手法である.
多点間の移動を繰り返すようなタスクに適しているが,
狭い通路(Narrow Passage)の通過が困難であるという問題が知られている.


\subsection{ポテンシャル法とナビゲーション関数}
グラフ探索やサンプリング手法が「経路(Path)」という離散的な線の集合を出力するのに対し,
環境全体にベクトル場(Vector Field)を定義し,その流れに乗って移動するアプローチが存在する.

\subsubsection{人工ポテンシャル法 (Artificial Potential Fields)}
Khatib\cite{khatib1986real}によって提案された人工ポテンシャル法(APF)は,
ロボットを仮想的な荷電粒子とみなし,
ゴールからの「引力」と障害物からの「斥力」を合成したポテンシャル場を構築する手法である.
ロボットはポテンシャルの勾配(Gradient)に従って最もエネルギーが低い方向へ移動するだけでよいため,
計算負荷が非常に軽く,リアルタイムな障害物回避に適している.
しかし,APFには重大な欠点がある.
それは「局所解(Local Minima)」の問題である.
U字型の障害物などに遭遇した場合,引力と斥力が釣り合ってしまい,
ゴールに到達する前に極小値で停止してしまう現象が発生する.

\subsubsection{ナビゲーション関数 (Navigation Functions)}
局所解の問題を解決するために,KoditschekとRimon\cite{koditschek1990robot}は
「ナビゲーション関数(Navigation Function)」の概念を提唱した.
これは,幾何学的な構成空間において,ゴールのみを唯一の大域的最小点(Global Minimum)とし,
その他すべての停留点が不安定な鞍点(Saddle Point)となるように
設計された特殊なポテンシャル関数である.
ナビゲーション関数が構築できれば,その勾配に従うだけで,
ロボットはどのような初期位置からでも必ずゴールへ到達できることが数学的に保証される.

グリッドマップ上におけるナビゲーション関数の実装例として,
KonoligeのGradient Method\cite{konolige2000gradient}が挙げられる.
これは,Dijkstra法や波及法(Wavefront Propagation)を用いて
各グリッドセルからゴールまでの「真の距離コスト」を計算し,
これをポテンシャルとみなすものである.
こうして得られた場は局所解を持たず,大域的に最適な経路情報を内包している.


\subsection{価値反復法(Value Iteration)}
これらの決定論的な大域経路計画に対し,
本研究で取り扱う価値反復法は,環境内のすべての状態(位置と向き)と行動を対応付けた
方策を計算するアプローチである.
この手法は,環境の不確実性を確率的に扱うことが可能であり,
外乱に対してロバストなナビゲーションを実現する.


\subsubsection{マルコフ決定過程(MDP)による定式化}
価値反復法を理解するためには,その基礎となるマルコフ決定過程(Markov Decision Process: MDP)
の定義が必要である.
MDPは,エージェントが環境と相互作用しながら学習・行動決定を行うための数理モデルであり,
以下の4つの要素の組 
$\langle \mathcal{S}, \mathcal{A}, \mathcal{P}, \mathcal{R} \rangle$ で定義される.

% \begin{enumerate}
%     \item \textbf{状態集合 $\mathcal{S}$ (State Space)}:
%     ロボットが取り得るすべての状態の集合.2次元グリッドマップ上での経路計画の場合,
% 	各グリッドセル $(x, y)$ が一つの状態 $s \in \mathcal{S}$ に対応する.
% 	さらに,ロボットの方位 $\theta$ を含めて $(x, y, \theta)$ を状態とすることもある.
% 
%     \item \textbf{行動集合 $\mathcal{A}$ (Action Space)}:
%     各状態でロボットが選択可能な行動の集合.
% 	グリッドマップ上では,隣接する8近傍(上下左右+斜め)への移動や,
% 	その場での停止などが行動 $a \in \mathcal{A}$ となる.
% 
%     \item \textbf{遷移確率 $\mathcal{P}_a(s, s')$ (Transition Probability)}:
%     状態 $s$ で行動 $a$ を選択したときに,次の時刻に状態 $s'$ へ遷移する確率.
%     \begin{equation}
%         \mathcal{P}_a(s, s') = \Pr(S_{t+1}=s' \mid S_t=s, A_t=a)
%     \end{equation}
%     決定論的な環境(A*などが想定する世界)では,ある行動を行えば100\%意図した隣接セルへ移動する.
% 	しかし実環境では,タイヤのスリップや制御誤差により,意図したセルへ移動できない場合がある.
% 	MDPではこの不確実性を確率分布として明示的にモデル化できる.
% 	例えば,「前進」を選択しても,10\%の確率で「横滑り」する,といった表現が可能である.
% 
%     \item \textbf{報酬関数 $\mathcal{R}_a(s, s')$ (Reward Function)}:
%     状態遷移に伴って得られる即時報酬(またはコスト).
% 	経路計画問題においては,通常「コストの最小化」または「負の報酬の最大化」として定式化される.
%     例えば,ゴール状態に到達したときに大きな正の報酬を与え,
% 	障害物に衝突したときに大きな負の報酬(ペナルティ)を与える.
% 	また,移動にかかる時間やエネルギーを表現するため,
% 	各ステップごとにわずかな負の報酬(ステップコスト)を与える.
% \end{enumerate}





\subsection{価値反復アルゴリズム}
ベルマン最適方程式 (\ref{eq:bellman_opt}) を用いて,
反復計算により $V^*(s)$ を求める手法が価値反復法(Value Iteration)である.
これはポテンシャル場に似ているが,ポテンシャル法が抱える局所解(Local Minima)の問題
(ゴール以外の窪みにハマって出られなくなる現象)が発生しないという強力な数学的保証がある.

%\dummyfig{価値関数の伝播と収束プロセスの可視化}{6cm}


\subsubsection{不確実性の考慮と確率的MDP}
Thrunらの著書『Probabilistic Robotics』\cite{thrun2005}において,
センサノイズやアクチュエータ誤差を考慮したMDPベースのプランニングの重要性が説かれている.
例えば,狭い通路を通る際,A*アルゴリズムなどの幾何的な最短経路探索では,
壁ギリギリを通るルートを選択しがちである.
しかし,実ロボットには移動誤差があるため,壁に衝突するリスクが高い.
確率的な遷移モデル $\mathcal{P}$ を組み込んだ価値反復法を用いると,
壁際での移動は衝突して大きなペナルティを受ける確率がある行動として評価される.
その結果,多少遠回りであっても,壁から距離を取った安全な広い通路を選択するような,
リスク回避的な経路(Risk-Aware Path)が自然に生成される.
これは,安全性が最優先されるサービスロボットにおいて極めて重要な特性である.


\subsubsection{計算コストの問題と高速化手法}
価値反復法の最大の欠点は,計算コストである.
状態空間 $\mathcal{S}$ のサイズに対して計算量が線形
(1回の反復あたり $O(|\mathcal{S}||\mathcal{A}|)$)に増加する.
広大な環境を高い解像度でグリッド化すると,状態数は数百万から数千万に達する.


上田らは,将来的には,計算機の性能が向上し,この計算コストの問題が解決されるとして
現在の移動ロボットで使われるミドルウェアのROS上で実装した.


%これに対し,いくつかの高速化手法が提案されている.
%\begin{itemize}
%    \item \textbf{Prioritized Sweeping}: 価値の変化が大きかった状態の周辺のみを優先的に更新する手法.
%    \item \textbf{GPUによる並列化}: 各状態の更新計算は独立性が高いため,
%	GPU(Graphics Processing Unit)を用いた並列計算と相性が良い.
%	近年ではCUDAを用いた実装により,数百万セルの更新を数ミリ秒で行う研究も報告されている.
%    \item \textbf{階層化(Hierarchical MDP)}: 環境を粗いトポロジカルな領域に分割し,
%	上位層で大まかな遷移を決定し,下位層で詳細な価値反復を行う手法.
%\end{itemize}


% \subsubsection{深層強化学習との融合}
% 近年では,Tamarらが提案したValue Iteration Networks (VIN)\cite{tamar2016value}のように,
% 価値反復の計算プロセス自体を微分可能なニューラルネットワークの層として組み込む研究が盛んである.
% これにより,生のセンサ画像(入力)から,障害物のコストマップ(中間表現),
% そして最適な移動方向(出力)までをEnd-to-Endで学習することが可能となる.
% これは,未知の環境における探索能力や,
% 人混みの中でのナビゲーションといった複雑なタスクにおいて成果を上げている.

\subsection{本研究の目的と構成}

\subsubsection{解決すべき課題}
以上の従来研究の調査から,価値反復法は移動ロボットの大域経路計画において
大域的最適性の保証,局所解の回避,不確実性への対処という優れた特性を持つことが確認された.
しかしながら,実環境での運用を考えた場合,以下の課題が依然として未解決,あるいは改善の余地がある.

\begin{enumerate}
    \item \textbf{動的環境へのリアルタイム追従性}: 環境の変化(ドアの開閉,荷物の移動など)に対し,
	全状態の価値関数を再計算するには時間がかかる.
	部分的な更新で整合性を保つ効率的なアルゴリズムが必要である.
    \item \textbf{コスト関数の設計困難性}: 安全性や社会的受容性(人への配慮)
	といった抽象的な指標を,どのような報酬関数として設計するかは自明ではない.
	逆強化学習(Inverse Reinforcement Learning)などのアプローチもあるが,計算負荷が高い.
    \item \textbf{3次元空間への拡張}: ドローンや不整地走行ロボットへの適用を考えた際,
	状態空間の次元爆発をどう抑えるかが課題となる.
\end{enumerate}


\section{研究目的}
価値反復ROSパッケージを用いたナビゲーションにおいて,
走り出しまでの時間を短縮することで
移動にかかる時間を短縮することを目的とする.


\section{論文の構成}
章では,問題設定と価値反復アルゴリズム,A*アルゴリズムについて述べ,
章では,提案手法,章では,実験について述べる.
章でまとめる.

